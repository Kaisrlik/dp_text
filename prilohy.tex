
\bibchap

\usebbl/c mybase


\label[zadani]
\app Zadání práce

\picw=\hsize % obr na sirku
\cinspic{images/802_miireg.pdf}
\nextoddpage


\app Přílohy


\picw=12cm % obr na sirku
\cinspic{images/des1.pdf}
\clabel[zapojeni1]{Schéma zapojení desky, konkrétně kontroléru Asix AX88772b.}
\caption/f Obrázek popisující schéma zapojení kontroléru Asix AX88772b. Zapojení poskitnuto firmou Retia a.s.~\cite[RETIA].

\picw=\hsize % obr na sirku
\cinspic{images/des2.pdf}
\clabel[zapojeni2]{Schéma zapojení desky, konkrétně přepínače Marvell 88e6065.}
\caption/f Obrázek popisující schéma zapojení přepínače Marvell 88e6065. Zapojení poskitnuto firmou Retia a.s.~\cite[RETIA].

\picw=\hsize % obr na sirku
\cinspic{images/802_controlreg.pdf}
\clabel[MIIcntr]{Kontrolní registr.}
\caption/f Obrázek popisující kontrolní registr STA rozhraní.   Obrázek je převzat z~\cite[IEEE8023].

\picw=\hsize % obr na sirku
\cinspic{images/802_statusreg.pdf}
\clabel[MIIstat]{Status registr.}
\caption/f Obrázek popisující stavový registr STA rozhraní.   Obrázek je převzat z~\cite[IEEE8023].

%\input kod


%\label[zkratky]
%\app Zkratky a symboly

%\font\mflogo=logo10 at11pt
%\def\METAFONT{{\mflogo METAFONT}}
%\def\METAPOST{{\mflogo METAPOST}}

%Tento text je až na výjimky převzat z~\cite[zyka].
%Tento text je až na výjimky převzat z~\cite[IEEE8023].

%\sec Zkratky

%Jako příklad pro popis zkratek poslouží pojmy ze světa \TeX{}u.

%\medskip
%\bgroup \leftskip=6.3em
%\abbrv[\TeX{}]  Program na přípravu elektronické sazby vysoké kvality
%   vytvořený Donaldem Knuthem. Program zahrnuje interpret makrojazyka.
%   Název programu se vyslovuje \uv{tech}.
%\abbrv[\METAFONT{}] Program a makro jazyk pro generování fontů
%   z vektorového do bitmapvého formátu vytvořený Donaldem Knuthem.
%\par\egroup



