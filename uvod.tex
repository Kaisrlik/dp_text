\chap Úvod

% komentare 
aaaaa

\chap Universal Serial Bus

\label[usb]

Universal Serial Bus (USB) je standart vyvyjeny od roku 1994 pro pripojeni periferii k pocitaci.
USB melo nahradit seriove a paralerni porty a vsak v soucasne dobe nahrazuje take PS/2, audio, video a take umoznuje prenaset ethernetove ramce.
USB je asymetricka topoligii pripojena do hvezdicove topologie - konkretne stromu. Pricemz USB huby vytvareji jednotlive uzly tohoto stromu a zarizeni tvori jejich listy. Maximalni hloubka toho stromu je 5 a k usb muze mit maximalne 127 zarizeni.
USB komunikace je zalozena na logickych kanalech tzv. pipes. Kazdemu vystupnimu kanalu by mel odpovidat jeden vstupni.
USB muze mit maximalne 30 endpointu. Tyto endpointy jsou inicializovany v prubehu enumarace USB zarizeni, ktere probiha po kontrolnim kanalu 0, ktery maji vsechna zarizeni.
Krome kontrolniho prenosu existuje izochroni, prerusovany a blokovy prenos.

\begitems
*Kontrolni - Slouzi ke konfiguraci zarizeni. Kazde zarizeni disponuje timto endpointem. Tento endpoint slouzi k enumeraci
*Izochroni - Slouzi  ke stalemu prenosu vetsich dat. Je garantovana latence, ale neni garantovao doruceni. Velmi vhodne pro audio a video.
*Prerusovany - urceny pro caste prenosy maleho mnozstvi dat. Garantovano doruceni a sircka pasma. Je defonovano maximalni zpozdeni
*Blokovy - je urcen pro prenos velkeho mnozstvi dat, ale nezarucuje pasmo ci latenci(prenos dat) viz prednaska Novaka
\enditems

Enumerace je posloupnost standardizovanych prikazu, ktery zapocal host. V prubehu enumerace se predavani tzv. deskriptory. Ktere obsahuji tyto dulezite informace o zarizeni. Na zaklade informace v deskriptorech se na zarizeni pouzije specialni ovladac.
Tento standart se v soucasne dobe rozvinul u umoznuje pripojit temer jakekoli zarizeni. Jako jsou ukazovaci zarizeni, pametova zarizeni, tiskarny, komunikacni zarizeni a mnohe dalsi.
Tato zarizeni jsou roztrizena do trid z dovodu s jednoceni druhu chovani jako jsou HID, PID Printer, Mass Storage a mnohe dalsi.
Toto sjednocene chovani ma za dusledek spolecnych ovladacu, ktera tato zarizeni mohou vyuzivat.
A vsak nektera z techto zarizeni se do techto trid nevnestnaji ci maji rozsirene chovani a potrebuji specailni pristup.

\chap Media Independent Interface

\sec popis

Media Independent Interface (MII) je specifikovano standardem IEEE 802.3 v kapitole 22. Interface je nezavysli na procesoru a fyzickem mediu.
MII propojuje dve vrstvy ISO/OSI a to vrstvu Data Link, konkretne jeji cast Medai Access Constroll (MAC) a vrstvou fyzickou (PHY).
MII obsahuje dva druhy sbernic a to datovou a manezovaci.
Menagement Data Input/Output (MDIO) Serial Management Interface (SMI) se používá k nastovování mezi MAC a PHY.

\medskip 
\clabel[MIItoISO]{Spojitost mezi MII a OSI/OSI}
%\picw=13cm \cinspic MIItoISO.png
\picw=6cm \cinspic a.png
\caption/f Obrazek popisuje spojistost mezi rozhranim MII, modelem OSO/OSI a modelem IEEE 802.3 CSMA/CD LAN \ref[IEEE8023].
\medskip

Na oprazku \ref[MIItoISO] muzeme videt pripojeni MII inerface na Reconciliation Sublayer (RS) a PCS/PLS. RS je podvrstva, ktera mapuje signali z MII na MAC/PLS obsluhu. Mapovani techto signalu muzeme videt na obrazku \ref[RStoMII]. PLS/PLC vrstvy, ktere se staraji o kodovani signalu.

%asi zbytecny detaili
\medskip 
\clabel[RStoMII]{Mapovani signalu RS to MII}
%\picw=13cm \cinspic MIItoISO.png
\picw=6cm \cinspic a.png
\caption/f Obrazek popisuje mapovani signalu mezi RS vstupem a vystupem a STA v MII \ref[IEEE8023].
\medskip

Az 32 PHY jednotek muze byt obsluhovano jednim manazovacim  MII rozhranim.
MII podporuje dva datove toky a to je 10~Mb/s a 100~Mb/s. Funkcionalita je identicka na oboch datovych tokach a lisi se poze v nominalni frakvenci hodin.

MII disponuje nekolika druhy signalu, ktere muzeme videt na prave strane obrazku \ref[RStoMII]. 

\begitems
*TX\_CLK -- Refenecni hodiny pro TX\_EN, TXD a TX\_ER signaly z RS do PHY. Zdrojem TX\_CLK je PHY. Frekvence hodin by mela byt 20\% z nominalniho prenosu dat +- 100ppm.
*RX\_CLK -- Refenecni hodiny pro RX\_DV, RXD a RX\_ER signaly z PHY do RS. Zdrojem RX\_CLK je PHY. RX\_CLK muze mit referenci hodnotu z prijimanych dat nebo muze byt odvozena z nominalni hodnoty jako u TX\_CLK.
*
\enditems


MII comprises of 4-bit transmit, TXD[3:0], and receive data, RXD[3:0], buses operating from independent 25MHz clocks TXCLK (local clock) and RXCLK (recovered line clock), respectively.
Additionally, on the transmit side, TXEN indicates when data is sent by the MAC and TXER indicates if errors have occurred during transmission.
Similarly, the receive side uses RXDV to indicate valid data and RXER to signal if physical layer errors are detected.
For half duplex operation signal COL indicates a collision during transmission, and CRS signals the presence of data  transmission and/or data reception.



% Ne do DP
%\sec GMII

%The Gigabit Media Independent Interface (GMII) is similar to the MII. The GMII uses the MII management interface and register set specified in 22.2.4. These common elements of operation allow Station Management to determine PHY capabilities for any supported speed of operation and configure the station based on those capabilities. In a station supporting both MII and GMII operation, configuration of the station would include enabling either the MII or GMII operation as appropriate for the data rate of the selected PHY. Most of the MII and GMII signals use the same names, but the width of the RXD and TXD data bundles and the semantics of the associated control signals differ between MII and GMII operation. The GMII transmit path clocking also differs significantly from MII clocking. MII operation of these signals and clocks is specified within Clause 22 and GMII operation is specified within Clause 35.

\sec RMII

RMIITM Specification (Reduced MII). RMII provides independent 2-bit wide transmit and receive paths synchronised to a common 50MHz reference clock. Furthermore, carrier sense and receive data valid are now combined  as one signal CRS\_DV. Collision detection is regenerated on the MAC side by ANDing signals TX\_EN and CRS\_DV. This reduces the total pins per port to 8 from 16 (MII)


\sec komunikace

\sec regspace

\sec linux?

\sec colibri t20

\secc arm arch



\secc perif

\sec asix

\secc popis

\sec marvell

\secc popis

\sec linux

\secc Co to je Ovladac?

\secc info kernel

\secc loading proces

\sec disribuce

\secc openwrt

\sec linux sub

\secc miibus

\secc device

\secc dsa

\secc usbnet

\secc sysfs

\sec Prakticka cast

\secc vytvor

\secc hw

\secc zvolen RMII na misto MII

\secc sw

\secc testing

\secc results

\secc Otestujte implementované řešení zejména s ohledem na propustnost sítě a stabilitu

\chap Zaver


\chap Zdroje:

http://www.usb.org/developers/docs/
MII,RMII,GMII,RGMII,SGMII,QGMII,XAUI
