\chap Úvod

Tu se deje kupa zajimavych veci ....

V kapitole pouzitech technologie \ref[pouztech] se dozvime teorii o XXX.


\sec Aktualni stav problematiky


%\chap Navrh

\label[pouztech]

\chap Použité technologie

Tato kapitola je rozebírána použitá technologie.
V první části je rozebrána sběrnice USB.
Další významnou sekcí je rozhraní Media Independet Interface, který předpokládá znalost modelu OSI, který je tu též uveden.
Ke konci této kapitoly bylo rozepsáno jak vypadájí ovladače v linuxu a tvář linuxového subsystému Distributed Switch Architecture a k němu přilehlé subsystémy.

\label[USB]

\sec Universal Serial Bus


Universal Serial Bus (USB) je standard vyvíjený od roku 1994 pro připojení periferií k pocitaci.
Konkrétně tento standard měl nahradit pomalé sběrnice jako jsou seriová linka, paralelní port, PS2 a mnohé další jednou sběrnicí \cite[LDD3].
V dnešní době USB umožňuje přenášet velké množství dat s rychlostí až 640 MB/s u USB 3.0 zařízení.
Díky tomu se USB rozrostlo a nyní podporuje skoro každý druh zařízení, která umožňují přenašet video, audio a dokonce i ethernetové rámce a pakety.
Tento standart se v současné době rozvinul a umožňuje připojit téměř jakékoli zařízení.
Jako jsou polohovací zařízení, paměťová zařízení, tiskárny, komunikační zařízení a mnohé další.
Tato zařízení jsou roztřízena do tříd z důvodu sjednocení funkcionality a chovaní zařízení jako jsou HID, PID Printer, Mass Storage a mnohé další.
Toto sjednocené chování má za důsledek společných ovladacu, ktera tato zarizeni mohou vyuzivat.
A vsak nektera z techto zarizeni se do techto trid nevnestnaji ci maji rozsirene chovani a potrebuji specailni pristup.

\sec Rysy USB zařízení


USB je deterministická sběrnice (Master/Slave), která podporuje detekci připojení (plug\&play) mechanismus a automatickou konfiguraci (hotswap).
Zařízení obsahuje 3 druhy zařízení \cite[USB]:
\begitems
*Hostitel -- je jediný v systému, řídí komunikaci a obvykle integruje rozbočovač označen jako kořenový.
Přiděluje zařízením unikátní adresu v síti.
*Rozbočovač -- distribuuje datové toky a identifikuje připojejí a odpojení dalších zařízení
*Zařízení -- koncové zařízení s požadovanou funkcionalitou, která odpovídají na kontrolní zprávy.
\enditems

USB je asymetrická sběrnice s jedním zařízením typu Master, která spíše vypadá jako strom složený z linek bod-bod.
Přičemž USB huby vytvářejí jednotlivé uzly tohoto stromu a zařízení tvoří jejich listy.
Linky USB sběrnice jsou 4 vodičové obsahující diferenciální pár datových linek, napěťový a zemnící vodič, které připojují zařízení a USB huby.
USB umožňuje připojit až 127 zařízení v rámci jedné USB sítě o hloubce maximálně pěti rozbočovaču.

Hostitel se a pta se kazdeho usb jestli nema neco na odelani.
Díky topologii nemuže USB vysílat aniž by bylo dotázáno. A proto je zde hostitel, který se ptá každého zařízení zda-li chce komunikovat a má nějaká data k odeslaní.
A proto USB umožňuje jednoduchý mechanismus detekce a enumerace za běhu systému, který je řízen a konfigurován automaticky hostitelem.
Průběh konfigurace se nazývá enumerace vice informací níže \ref[USBkom].

\label[USBkom]

\sec Komunikace

USB komunikace je založena na logických kanálech tzv. rourách. Každému výstupnímu kanálu by měl odpovídat právě jeden vstupní.
Koncový bod je roura s definovaným směrem.
USB muže mít maximalně 30 těchto koncových bodů. Tyto koncové body jsou inicializovány v průběhu enumarace USB zarizeni, které probíha po kontrolním rouře označenou číslem 0, kterou mají všechna zařízení společnou.
Každý koncový bod obsahuje rouru s předem definovanym typem přenosu popsanou níže.

\begitems
*Řídící -- Obousměrná roura sloužící ke konfiguraci zařízení. Každé zařízení disponuje tímto druhem roury. Má rezervovanou určitou přenosovou kapacitu.
*Izochronní - Jednosměrná roura sloužící ke stálému přenosu většího objemu dat. Má garantovanou latenci, avšak přenos není spolehlivý. Tento typ roury je velmi vhodný pro audio a video.
*Přerušovací - Jednosměrná roura sloužící pro časté přenosy malého množství dat. Má garantovano šířku pásma a přenos je spolehlivý. V případě chyby se přenos opakuje.
*Blokový - Jednosměrná roura sloužící přenosu velkého množství dat. Nemá rezervovanou žádnou přenosovou kapacitu ani dobu odezvy. Komunikace po této rouře je spolehlivá a v případě chyby se přenos opakuje.
\enditems

Enumerace je posloupnost standardizovaných příkazů, který započal hostitel.
V průběhu enumerace se předávají deskriptory, které obsahují důležité informace o zařízení.
Na základě informací v deskriptorech se operační systém rozhoduje, který druh driveru přiřádí tomuto druhu zařízení.

%LDD3 pekny popis zarizeni http://www.makelinux.net/ldd3/?u=chp-13-sect-4


\label[MII]

\sec Media Independent Interface

Media Independet Interface je typ rozhraní, který umožňuje připojení nezávyslého na procesoru a k fyzickém médiu.
Media Independent Interface (MII) je specifikováno standardem IEEE 802.3 v kapitole 22 \ref[IEEE8023].
MII propojuje dvě vrstvy ISO/OSI a to vrstvu Data Link, konkretne jeji cast Medai Access Constroll (MAC) a vrstvou fyzickou (PHY).

\medskip
\clabel[MIItoISO]{Spojitost mezi MII a OSI/OSI}
%\picw=13cm \cinspic MIItoISO.png
\picw=6cm \cinspic a.png
\caption/f Obrázek popisuje spojistost mezi rozhranim MII, modelem OSO/OSI a modelem IEEE 802.3 CSMA/CD LAN \cite[IEEE8023].
\medskip

Na obrázku \ref[MIItoISO] můžeme vidět připojení MII na Reconsiliation Sublayer (RS) a PCS/PLS.
RS je podvrstva, která mapuje signály z MII na MAC/PLS obsluhu.
Mapování těchto signálů můžeme vidět na obrázku \ref[RStoMII].
PLS/PLC jsou vrstvy, které se starají o kódování a dekódování signálů.


%asi zbytecny detaili
\medskip
\clabel[RStoMII]{Mapování signálů RS to MII}
%\picw=13cm \cinspic MIItoISO.png
\picw=6cm \cinspic a.png
\caption/f Obrázek popisuje mapování signalů mezi RS vstupem a výstupem a také STA na MII \cite[IEEE8023].
\medskip

Až 32 PHY jednotek může být obsluhováno jedním menežovacim MII rozhraním.
MII podporuje dva datové toky a to je 10~Mb/s a 100~Mb/s.
Funkcionalita je identicka na obou datových tokách a liší se pouze v nominální frekvenci hodin.

Fyzické rozhraní MII se skládá ze dvou druhů sběrnic a to datové \ref[MIIdat] a menežovací \ref[MIImdio].

\label[MIIdat]

\secc Datová část MII rozhraní

Datová část MII disponuje několika druhy signálu, které mužeme vidět na pravé straně obrázku \ref[RStoMII] popsané níže.

\begitems
*TX\_CLK -- Pin vysílající refeneční hodinový signál pro synchronizaci pinů TX\_EN, TXD a TX\_ER, která vede z RS do PHY. 
Zdrojem TX\_CLK je PHY. 
Frekvence tohoto signálu by měl být 20\% z nominální hodnoty přenosu dat +- 100ppm.
*RX\_CLK -- Pin vysílající refeneční hodinový signál pro synchronizaci pinů RX\_DV, RXD a RX\_ER, která vede z PHY do RS.
Zdrojem RX\_CLK je PHY. 
RX\_CLK může mít referenční hodnotu z přijímaných dat nebo muže být odvozena z nominální hodnoty tak jako u TX\_CLK.
*TX\_EN -- Pin indikující, že RS je připravena odesílat data.
*TDX -- Čtveřice datových pinů (TDX[3:0]) ovládána RS, která přenáší data synchroně na základě TX\_CLK. 
TDX[0] je nejméně významný bit.
Pokud je TX\_EN fyzicky odpojen TDX nemá žádný efekt na PHY.
*TX\_ER -- Pokud PHY vysílá jeden či více symbolů, která nejsou součástí dat nebo předčasně dojde k přerušení rámce, tak tento pin synchroně na základě hodinového signálu TX\_CLK a nastaveného signálu TX\_EN vyhodí chybu.
*RX\_DV -- Pin označující, že přijatá data jsou validní. Tento pin je řizen PHY a je synchroní s RX\_CLK.
*RXD -- Čtveřice datových pinů (RDX[3:0]) ovladané PHY, které slouží k přenosu dat z PHY do RS. RXD[0] přenáší nejméně významný bit.
*RX\_ER -- Pin řízený RHY, který indikuje RS, že nastal error detekovaný někde v průběhu přijímaní rámce.
Pokud je RX\_DV odpojen nemá žadný vliv na RS.
*CRS -- Tento pin je aktivní pokud příjemce nebo odesilatel je zaneprázdněn a neaktivní pokud jsou idle.
Není synchroní jak na TX\_EN tak ani na RX\_EN.
*COL -- Tento pin vyvolán PHY pokud na mediu nastane kolize.
*CRS a COL -- Piny jous neaktivní pokud ve stavu idle. Nejsou synchroní na TX\_CLK tak ani na RX\_CLK. Nedefinovane chovani COL a CRS pokud bit 0.8 je aktivni.
%Reprezentuje pokud josu data na odeslani nebo prijeti.
\enditems

Prenos dat je popsan formatem
<inter-frame><preambule><začátek datového rámce><data><konec datového rámce>.
Pričemž byte je rozdělen na dvě poloviny tzv. nibble. A ty jsou odeslány ve dvouc částech po TXD. 
Jak mužeme vidět na obrázku X.
%popis  <inter-frame><preamble><sfd><data><efd>  ?

\medskip
\clabel[transiotionMII]{Komunikace bez kolize po MII}
%\picw=13cm \cinspic MIItoISO.png
\picw=6cm \cinspic a.png
\caption/f Obrazek ukazuje jak vypada komunikace po MII bez zadne kolize \cite[IEEE8023].
\medskip

\secc Menežovací rozhraní a příslušný registrový prostor

Menežovací čast sběrnice označena STA používá k nastavování a čtením hodnot mezi MAC a PHY.
Ta obsahuje dva datové vodiče:
\begitems
*MDC -- Pin určující časovou základnu pro MDIO signal. MDC je aperiodický signál, který nemá žádnou minimální ani maximální frekvenci, avšak by se měl pohybovat v rozmezí od 160 ns do 400 ns.
*MDIO -- Pin, který je vstupně výstupní a posílají se po něm signály oboumi směry je proveden třístavovým obvodem mezi PHY a STA.
Používá se k přenosu kontrolních zpráv mezi těmito zařizeními. MDIO je synchroní s MDC a informace jsou řizeny z PHY.
\enditems

Manezovaci rozhrani je specifikovano dvou linkovym seriovym rozhranim pripojenym k menozovacim vstupum PHY pro ovladni PHY a zbirani informaci o PHY.
MII obsahuje dva zakladne sety registru Controlni a Stavovy. Vsechny PHY, ktere podporuji MII by meli zahrnovat tuto sadu registru. Stavovy a Controlni registrove prostory muzete videt v priloha X. A specifikuji zakladni vlastnosti pro 100Mb/s 1Gb/s PHYs. Registry 2-14josu casti rozsireneho regitroveho prostoru.(22.2.4 info o reg)
Format menezovacich ramcu je popsan tabulkou tab X a je odesilan z leva do prava.

\midinsert \clabel[MDIOframe]{Format menezovaciho ramce.}
\ctable{lllllllll}{
 \hfil   operace  & PRE & ST & OP & PHYAD & REGAD & TA & DATA & IDLE \crl \tskip4pt
          cteni   & 1...1 & 01 & 10 & AAAAA & RRRRR & Z0 & D...D & Z \cr
          zapis   & 1...1 & 01 & 01 & AAAAA & RRRRR & 10 & D...D & Z \cr
}
\caption/t Format menezovacich zprav odesilany z leva do prava \cite[IEEE8023] (table 22-12).
\endinsert

V tabulce \ref[MDIOframe] muzeme videt hodnotu Preambule (PRE), ktera je posloupnost 32bitu hodnoty logicke 1 slouzici k synchronizaci s MDC.
Zacatek ramce (ST) je 01. Pote nasleduje Operacni kod (OP), ktery urcuje druh operace cteni nebo zapisu. PHYAD je fyzicka adresa zarizeni o delce 5bitu tj. az 32 zarizeni. Prvni bit je MSB. Dalsi polozkouy je adresa registru (REGAD) cislovana jako vyse. Dalsim blokem je TA coz slouzi u cteni k prohozeni odesilatele, kdy se aktivni odesilatel prehodi do stavu vysoke impedance a necha vysilani PHY. Pri zapisu je vysilana logicka 1. Data jsou 16bitova hodnota.


% Ne do DP
%\sec GMII

%The Gigabit Media Independent Interface (GMII) is similar to the MII. The GMII uses the MII management interface and register set specified in 22.2.4. These common elements of operation allow Station Management to determine PHY capabilities for any supported speed of operation and configure the station based on those capabilities. In a station supporting both MII and GMII operation, configuration of the station would include enabling either the MII or GMII operation as appropriate for the data rate of the selected PHY. Most of the MII and GMII signals use the same names, but the width of the RXD and TXD data bundles and the semantics of the associated control signals differ between MII and GMII operation. The GMII transmit path clocking also differs significantly from MII clocking. MII operation of these signals and clocks is specified within Clause 22 and GMII operation is specified within Clause 35.
%GMII obsahuje 3ti zakladni registr Extended status reg(15)

\secc RMII

RMIITM Specification (Reduced MII). RMII provides independent 2-bit wide transmit and receive paths synchronised to a common 50MHz reference clock. Furthermore, carrier sense and receive data valid are now combined  as one signal CRS\_DV. Collision detection is regenerated on the MAC side by ANDing signals TX\_EN and CRS\_DV. This reduces the total pins per port to 8 from 16 (MII)

% TODO
%\sec linux

%\sec info kernel

%\sec loading proces

%\sec disribuce

%\secc openwrt

\sec linux sub

\secc Co to je Ovladac?

Ovladac je kus programu ktery operuje nebo controluje urcite zarizeni pripojene k pocitaci. Ovladace vytvareni softwareove rozhrani pro hardwarove zarizeni, ktere umoznuji operacnimu systemy ovladat tento hardware aniz by vedel jak presne funguje.

Ovladace zarizeni maji v linuxu specialni roly \ref[LDD3]. Tyto kusy kodu umoznuji HW aby odpovidal definovanym rozhranim.

\secc miibus

\secc device

At the lowest level, every device in a Linux system is represented by an instance of struct device. The device structure contains the information that the device model core needs to model the system. Most subsystems, however, track additional information about the devices they host. As a result, it is rare for devices to be represented by bare device structures; instead, that structure, like kobject structures, is usually embedded within a higher-level representation of the device \ref[DEV].

\secc Distributed Switch Architecture

 Distributed Switch Architecture (DSA) je protokol pro menezovani hardwarovych switchu, ktere obsahuji MII menezovaci registry a prikazy nastavujici switche a ethernotovy format ktery signalizuje, na kterm portu apyl paket prijet nebo ktery se hodla odelsat \ref[DSA].
Tento driver podporuje swtihe, ktere jsou pripojeny zpusobem:

%+-----------+       +-----------+ \hfill \break
%|           | RGMII |           |
%|           +-------+           +------ 1000baseT MDI ("WAN")
%|           |       |  6-port   +------ 1000baseT MDI ("LAN1")
%|    CPU    |       |  ethernet +------ 1000baseT MDI ("LAN2")
%|           |MIImgmt|  switch   +------ 1000baseT MDI ("LAN3")
%|           +-------+  w/5 PHYs +------ 1000baseT MDI ("LAN4")
%|           |       |           |
%+-----------+       +-----------+

Tento switch driver reprezentuje kazdy port jako oddeleny sitove rozhrani. Pollovani, stav portu, MII manazement interfacu je proveden diky ethtoolu.


DSA podporuje i propojeni mezi switchy, ktere pak muzem take ovladat.

%        +-----+          +--------+       +--------+
%        |     |eth0    10| switch |9    10| switch |
%        | CPU +----------+        +-------+        |
%        |     |          | chip 0 |       | chip 1 |
%        +-----+          +---++---+       +---++---+
%                             ||               ||
%                             ||               ||
%                             ||1000baseT      ||1000baseT
%                             ||ports 1-8      ||ports 9-16





\secc usb driver

Linuxove jadro podporuje dva druhy zarizeni drivers on a host system and drivers on a device \cite[LDD3]. USB ovladac na hostosky system ke kteremu je USB zarizeni pripojeno.
Ovladace na zarizeni neboli \"USB gadget drivers\" ovladini zarizeni vypadajici jako hostovska stanice pripojena prez USB.
NA obrazku nize muzeme videt USB stack. Kde USB muze zit v nekolika ruznych subsystemych (net, block, char ...). USB core dava rozhrani pro USB ovladace, kteri chteji kontrolovat a pristupovat k HW. Ruzne druhy kontroleru jsou reprezentovany v systemu.

USB endpointy jsou popsany v kernelu strukturou struct usb\_host\_endpoint a ta obsahuje informace o realnem endpintu v strukture struct usb\_endpoint\_descriptor - ten popisuje data pomoci prijmutych dekriptoru.

USB interface nekolik endpointu, ktere tvori jedno logicke pripojeni jako je mys, klavesni, video atd. Nektera USB zarizeni mohou mit vice interfacu. USB speaker ma dva interfaci jako je USB keyboear pro tlacitka a USB audio stream.%TODO presunout nahoru
Pro kazdy interface Linux pouziva jeden hardwarovy ovladac. A tento interface je popsan struktorou struct usb\_interface. Tato struktura je to co USB core dava USB driverum.
Kazde zarizeni je vazano konfiguraci a jedno USB zarizeni muze mit vice configuraci, ktera se mohou menit. A vsak Linux nemuze obsluhovat vice configuraci v jeden cas. Linux popisuje configuraci v struct usb\_host\_config a cele USB zarizeni strukturou struct usb\_device.
USB drivery obycejne prepisuji data z usb\_interafacu do struct usb\_device.

Samotny usb\_device a usb\_interface je zobrazeny v sysfs jako jednotliva zarizeni.

usb\_device:
/sys/devices/pci0000:00/0000:00:09.0/usb2/2-1

usb\_interface je pojmenovan podle formatu root\_hub-hub\_port:config.interface(pro hlubsi stromy je schema root\_hub-hub\_port-hub\_port:config.interface) :
/sys/devices/pci0000:00/0000:00:09.0/usb2/2-1/2-1:1.0

Pro komunikaci se vsemi USB zarizenimi linux pouziva USB request block (urb). Urb se pouziva k posilani a prijmu da pro specificky usb endpint a specifivcke usb zarizeni v asynchronim zpusobem.

%\secc net\_device

\secc usbnet

Usbnet je subsystem linuxoveho jadra, ktery umoznuje ovladat USB-sitove zarizeni jako je ethernet, DSL, IDSN atd \ref[USBNET].
Je to genericky USB sitovy framework, ktery pracuje na ruznych rochlostech a ruznymi protokoli. 


\secc System Filesystem

SYstem FileSystem (Sysfs) je charakteristika linuxu 2.6, ktera umozuje kernelovemu kodu exportovat informace do userspacu za pouziti pameti za pouziti VFS\ref[SYSFS]. Vetsinou kazdy file vetsinou ASCII obsahuje pouze jednu hodnotu.
Hlavnim ucelem je reprezentovat objekty, jejich atributy a jejich vstahy navzajem.
Poskytuje dve slozky:
*kernel programing interface -- pro export itemu skrz sysfs
*user interface -- k videni a manipulaci techto itemu, ktere mapuje zpet na objekty v kernelu
Mapovani mezi objekty muzeme videt v \ref[SYSFSmap].

\midinsert \clabel[SYSFSmap]{Mapovani objektu a atributu v sysfs.}
\ctable{ll}{
 \hfil  Interni   & Externi  \crl \tskip4pt
        Kernelove objekty    & Slozky  \cr
        Atributy objektu     & Soubory  \cr
        Vazby mezi objekty   & Symbolicke odkazy  \cr
}
\caption/t  Tabulka prevzana z \cite[SYSFS].
\endinsert

V systemu muzeme videt tyto zarizeni v slozce /sys a muzeme tam napriklad videt tyto zarizeni ....

%/sys/
%|-- block
%|-- bus
%|-- class
%|-- devices
%|-- firmware
%|-- module
%‘-- power


\chap Pouzity hardware

\sec Colibri T20

Colibri T20 je SODIMM pocitacovy modul posltaven na zaklade NVIDIA Tegra 2 embeded systemu na cipu (SOC). 
SOC modul Tegra 2 je zalozanan na dvoujadrovem procesoru Cortex A9 symmetric MPCore od firmy ARM s rychlosti okolo 1 Ghz.
Jelikoz se jedna o SOC obsahuje tento cip take mnoho mluku jako Audio/Video rekorder. Graficky cip s podporou 2D rendrovani a 3D pixel a vektor shaderem.

Cip obsahuje periferie popsane na blokovem diagramu z 
%(https://www.toradex.com/computer-on-modules/colibri-arm-family/nvidia-tegra-2).

%chtelo by to swap Colibri a ARM ci spojit...
\sec ARM Cortex A9

ARM je 32-bitove RISC architektura vyvajena spolecnosti ARM Holding venujici se vyvoji procesorovych jader.

Dvoujadrovy procesor ARM Cortex A9 symmetric MPCore je jeden z aplikacnich procesoru.

Tento procesor porporuje out-of-order a spekulaticni provadeni instrukci. 
Ma plne koheretni podporu pro symetricke procesory.
Tento modulu dosahuje velmi vysokeho grafickeho ale take vypocetniho vykonu viz. \ref[testColibriT20]. 
Procesor tez obsaguje jednotku pro vypocty s plovouci radovou carkou.

\sec Asix AX88772b

Jako jediny z vyrobnich rad dosahuje v revizi AX88772bXX ma teplotni rozsahy od -40C az +85C.

AX88772b je USB 2.0 - 10/100 Mb/s Fast Ethernet kontroler s vysokou vykonosti a intergraci ASIC, ktera umoznuje malou zpotrebu a plug-and-play Fast Ethernet internetove pripojeni pro zarizeni pouzivajici stantardyzovane USB \ref[USB].

AX88772b podporuje volitelny Multy-Function-Bus(MFA/MFB) pro externi PHY nebo externim MAC. MFA/MFB muze byt nakonfigutovano jako General purpose I/O, RMII \ref[RMII] pro implementaci Fast ethernetu nebo jako Reverse-RMII pro MAC-to-MAC pripojeni k nejakakemu MCU s ethernet MAC RMII rozhranim.
%neudelat 802.3 chap?
MAC Core podporuje 802.3 a 802.3u MAC funkce, jako je prijem a odeslani ramce, kontrolu CRC, duplex mode, forwarding, flow-control, detekce kolizi atd.

%jake umoznuje propojeni po rmii?

\sec Marvell 88E6065

Obsahuje  petkrat yysilac na fyzicke vrstve 10~BASE-T/100~BASE-TX, ktere muzou byt pouzity take pro optickou linku 100~BASE-FX.
Sestym portem je nezavysli Fast ethernet media access controller.
Je jedno cipovy obvod integrujici 4 + 2 portu Fast Ethernetoveho switche. Toto zarizeni podporuje Quality of Service (QoS) A vyjmecne muzou vyuzivat podpory pro routry a vychozi brany.
Zarizeni ma high-speed, non-blocking four traffic class QoS. True plag-n-play je podporovano s Auto-Crossoer, Auto-polarity a Auto-nogotiation na fyzike vrtve. Zarizeni podporuje 64 z 4096 802.1Q WLAN s 3 levlovou ochranou. Ma dve RMII/MII/SNI rozhrani, ktera mohou byt pripojena k menezovacimu rozhrani nebo Router CPU s integrovanou MAC.

MAC a PHY rozhrani jsou plne kompatibilni se standardy IEEE~802.3, IEEE~802.3u a IEEE~802.3x.
Zarizeni muze byt configurovano prez STA (SNI) nebo moze nastaveni nacitat z flash.

%jake umoznuje porjojeni RMII

\sec deska USB prepinace

Jedna se o desku, ktera umoznuje konfiguraci ethernetoveho prepinace za pomoci USB kontroleru.
Deska obsahuje USB kontroler AX88772b od firmy Asix ke kteremu je pripojena externi pamet eeprom M93-C66.
A tento kontroler je pripojen RMII rozhranim k cipu Marvell 88E6065. Marvell 88E6065 je ethernetovy prepinac, ktery ma k sobe krome zmineneho RMII portu pripojeno pet ethernetovych rozhrani a taktez je knemu pripojena externi pamet eeprom.
Oba tyto cipy tez vlastni vlastni krystal na frakvenci 25~Mhz.
Deska je tez ossazena regulatorem napeti.

Deska USB prepinace je vyrobena firmou Retia a.s. na zaklade jejich navrhu a zapojeni popsanych v priloze \ref[zapojeni] \ref[zapojeni2] \ref[zapojeni3].

\chap Implementace

popis o co se tu jedna

\sec navrh

\sec Rozchozeni hardwaru

Rozchozeni hardwaru byla jedna z nejobtiznejsich casti.

Pricemz se zacalo s testovanim USB kontroleru Asix AX88772b. A po rozchozeni komunikacnich interfacu bylo pokracovano s ladenim ethernetoveho prepinace 88E6065.
Posledni casti bylo testovani ethernetovych portu.

Pricemz bylo objeveno nekolik chyb v navrhu zarizeni. Ktera na zaklade mojich pripominek byla opravena(XXX to zaveru).

\secc asixu

Kdy byla objevena zavada na pripojeni kontroleru.
Tato zavada se projevovala chybnou komunikaci na USB sbernici. Tu to chybnou komunikaci muzete videt na obrazku \ref[usbbuserr].

\medskip
\clabel[usbbuserr]{Chybna komunikace na usb sbernici.}
%\picw=13cm \cinspic MIItoISO.png
\picw=6cm \cinspic a.png
\caption/f Na obrazku muzeme videt chybnou komunikaci na usb sbernici mezi PC a usb kontrolerem Asix AX88772b zavinou spatnym nepripojenim blokovacich kondenzatoru.
\medskip

Po uprave zapojeni kontroleru. Zacal USB kontroler odpovidat na dotazy.

Dale bylo potreba nastavit vhodne chovani cipu Asix aby zacalo fungovat manazovaci rozhrani RMII sbernice. Bylo zvoleno zapojeni, ktere aktivuje RMII. Pote, jiz bylo mozne vycitat data ze zarizeni Marvell.

\secc marvell

Na zarezeni 88E6065 bylo provedeno nekolik zasahu. Byli pridany pull-up a pull-down rezistory pro spravne a deterministicke chovani prepinace.
Dale byl doplne kondenzator, ktery zarizeni vyresotoval po dobu konstanty RC po pripojeni pripinace k napajeni. Tento kro byl dulezity ponevadz vsechny zarizeni marvell chce vstupy pri startu v definovene hodnote. A stavalo ze pri zapnuti prepinace se hodnoty ruznych zapnuti ruzne menili.

\sec zpracovani

\sec celek?

\sec odezva komunity

Odezva komunity byla dobra - po odeslani RFC patche (*pozn pod carou) se mi dostalo odpovedi behem pouhych par hodin po jeho odesalni.

\chap Otestujte implementované řešení zejména s ohledem na propustnost sítě a stabilitu

Tato kapitola se zabiva

\sec propustnost

\sec porovnani s jinym eth interfacem

\sec konfigurace 

co se testuje a jak se testuje(co pisu za prikazy) -> vysledky

\sec logterm

\chap Zaver

funguje, odezva, linuxova komunita, dosazene vysledky, vykon
MII,RMII,GMII,RGMII,SGMII,QGMII,XAUI
