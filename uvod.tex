\chap Úvod
%TODO list
% usb obr, enumaration, mazna edpoint+interface
Tu se deje kupa zajimavych veci ....

V kapitole pouzitech technologie \ref[pouztech] se dozvime teorii o XXX.


\sec Aktualni stav problematiky


%\chap Navrh

\label[pouztech]

\chap Použité technologie

Tato kapitola je rozebírána použitá technologie.
V první části je rozebrána sběrnice USB.
Další významnou sekcí je rozhraní Media Independet Interface, který předpokládá znalost modelu OSI, který je tu též uveden.
Ke konci této kapitoly bylo rozepsáno jak vypadájí ovladače v linuxu a tvář linuxového subsystému Distributed Switch Architecture a k němu přilehlé subsystémy.

\label[USB]

\sec Universal Serial Bus


Universal Serial Bus (USB) je standard vyvíjený od roku 1994 pro připojení periferií k pocitaci.
Konkrétně tento standard měl nahradit pomalé sběrnice jako jsou seriová linka, paralelní port, PS2 a mnohé další jednou sběrnicí \cite[LDD3].
V dnešní době USB umožňuje přenášet velké množství dat s rychlostí až 640 MB/s u USB 3.0 zařízení.
Díky tomu se USB rozrostlo a nyní podporuje skoro každý druh zařízení, která umožňují přenašet video, audio a dokonce i ethernetové rámce a pakety.
Tento standart se v současné době rozvinul a umožňuje připojit téměř jakékoli zařízení.
Jako jsou polohovací zařízení, paměťová zařízení, tiskárny, komunikační zařízení a mnohé další.
Tato zařízení jsou roztřízena do tříd z důvodu sjednocení funkcionality a chovaní zařízení jako jsou HID, PID Printer, Mass Storage a mnohé další.
Toto sjednocené chování má za důsledek společných ovladacu, ktera tato zarizeni mohou vyuzivat.
A vsak nektera z techto zarizeni se do techto trid nevnestnaji ci maji rozsirene chovani a potrebuji specailni pristup.

\secc Rysy USB zařízení


USB je deterministická sběrnice (Master/Slave), která podporuje detekci připojení (plug\&play) mechanismus a automatickou konfiguraci (hotswap).
Zařízení obsahuje 3 druhy zařízení \cite[USB]:
\begitems
*Hostitel -- je jediný v systému, řídí komunikaci a obvykle integruje rozbočovač označen jako kořenový.
Přiděluje zařízením unikátní adresu v síti.
*Rozbočovač -- distribuuje datové toky a identifikuje připojejí a odpojení dalších zařízení
*Zařízení -- koncové zařízení s požadovanou funkcionalitou, která odpovídají na kontrolní zprávy.
\enditems

USB je asymetrická sběrnice s jedním zařízením typu Master, která spíše vypadá jako strom složený z linek bod-bod.
Přičemž USB huby vytvářejí jednotlivé uzly tohoto stromu a zařízení tvoří jejich listy.
Linky USB sběrnice jsou 4 vodičové obsahující diferenciální pár datových linek, napěťový a zemnící vodič, které připojují zařízení a USB huby.
USB umožňuje připojit až 127 zařízení v rámci jedné USB sítě o hloubce maximálně pěti rozbočovaču.

Hostitel se a pta se kazdeho usb jestli nema neco na odelani.
Díky topologii nemuže USB vysílat aniž by bylo dotázáno. A proto je zde hostitel, který se ptá každého zařízení zda-li chce komunikovat a má nějaká data k odeslaní.
A proto USB umožňuje jednoduchý mechanismus detekce a enumerace za běhu systému, který je řízen a konfigurován automaticky hostitelem.
Průběh konfigurace se nazývá enumerace vice informací níže \ref[USBkom].

\label[USBkom]

\secc Komunikace

USB komunikace je založena na logických kanálech tzv. rourách. Každému výstupnímu kanálu by měl odpovídat právě jeden vstupní.
Koncový bod je roura s definovaným směrem.
USB muže mít maximalně 30 těchto koncových bodů. Tyto koncové body jsou inicializovány v průběhu enumarace USB zarizeni, které probíha po kontrolním rouře označenou číslem 0, kterou mají všechna zařízení společnou.
Každý koncový bod obsahuje rouru s předem definovanym typem přenosu popsanou níže.

\begitems
*Řídící -- Obousměrná roura sloužící ke konfiguraci zařízení. Každé zařízení disponuje tímto druhem roury. Má rezervovanou určitou přenosovou kapacitu.
*Izochronní - Jednosměrná roura sloužící ke stálému přenosu většího objemu dat. Má garantovanou latenci, avšak přenos není spolehlivý. Tento typ roury je velmi vhodný pro audio a video.
*Přerušovací - Jednosměrná roura sloužící pro časté přenosy malého množství dat. Má garantovano šířku pásma a přenos je spolehlivý. V případě chyby se přenos opakuje.
*Blokový - Jednosměrná roura sloužící přenosu velkého množství dat. Nemá rezervovanou žádnou přenosovou kapacitu ani dobu odezvy. Komunikace po této rouře je spolehlivá a v případě chyby se přenos opakuje.
\enditems

Enumerace je posloupnost standardizovaných příkazů, který započal hostitel.
V průběhu enumerace se předávají deskriptory, které obsahují důležité informace o zařízení.
Každý druh deskriptoru opsahuje specifické informace.
%info o deskriptorech???
Jakou jsou ID produktu, ID vendora, XXX.
Na základě informací v deskriptorech se operační systém rozhoduje, který druh driveru přiřádí tomuto druhu zařízení.

%LDD3 pekny popis zarizeni http://www.makelinux.net/ldd3/?u=chp-13-sect-4
%MAC podporuje 802.3 a 802.3u MAC funkce, jako je prijem a odeslani ramce, kontrolu CRC, duplex mode, forwarding, flow-control, detekce kolizi atd.

\label[MII]

\sec Media Independent Interface

Media Independet Interface je typ rozhraní, který umožňuje připojení nezávyslého na procesoru a k fyzickém médiu.
Media Independent Interface (MII) je specifikováno standardem IEEE 802.3 v kapitole 22 \ref[IEEE8023].
MII propojuje dvě vrstvy ISO/OSI a to vrstvu Data Link, konkretne jeji cast Medai Access Constroll (MAC) a vrstvou fyzickou (PHY).

\medskip
\clabel[MIItoISO]{Spojitost mezi MII a OSI/OSI}
\picw=16cm \cinspic images/802_iso.pdf
\caption/f Obrázek popisuje spojistost mezi rozhranim MII, modelem OSO/OSI a modelem IEEE 802.3 CSMA/CD LAN \cite[IEEE8023].
\medskip

Na obrázku \ref[MIItoISO] můžeme vidět připojení MII na Reconsiliation Sublayer (RS) a PCS/PLS.
RS je podvrstva, která mapuje signály z MII na MAC/PLS obsluhu.
Mapování těchto signálů můžeme vidět na obrázku \ref[RStoMII].
PLS/PLC jsou vrstvy, které se starají o kódování a dekódování signálů.


%asi zbytecny detaili
\medskip
\clabel[RStoMII]{Mapování signálů RS to MII}
%\picw=13cm \cinspic MIItoISO.png
\picw=15cm \cinspic images/802_rs.pdf
\caption/f Obrázek popisuje mapování signalů mezi RS vstupem a výstupem a také STA na MII \cite[IEEE8023].
\medskip

Až 32 PHY jednotek může být obsluhováno jedním menežovacim MII rozhraním.
MII podporuje dva datové toky a to je 10~Mb/s a 100~Mb/s.
Funkcionalita je identicka na obou datových tokách a liší se pouze v nominální frekvenci hodin.

Fyzické rozhraní MII se skládá ze dvou druhů sběrnic a to datové \ref[MIIdat] a menežovací \ref[MIImdio].

\label[MIIdat]

\secc Datová část MII rozhraní

Datová část MII disponuje několika druhy signálu, které mužeme vidět na pravé straně obrázku \ref[RStoMII] popsané níže.

\begitems
*TX\_CLK -- Pin vysílající refeneční hodinový signál pro synchronizaci pinů TX\_EN, TXD a TX\_ER, která vede z RS do PHY. 
Zdrojem TX\_CLK je PHY. 
Frekvence tohoto signálu by měl být 20\% z nominální hodnoty přenosu dat +- 100ppm.
*RX\_CLK -- Pin vysílající refeneční hodinový signál pro synchronizaci pinů RX\_DV, RXD a RX\_ER, která vede z PHY do RS.
Zdrojem RX\_CLK je PHY. 
RX\_CLK může mít referenční hodnotu z přijímaných dat nebo muže být odvozena z nominální hodnoty tak jako u TX\_CLK.
*TX\_EN -- Pin indikující, že RS je připravena odesílat data.
*TDX -- Čtveřice datových pinů (TDX[3:0]) ovládána RS, která přenáší data synchroně na základě TX\_CLK. 
TDX[0] je nejméně významný bit.
Pokud je TX\_EN fyzicky odpojen TDX nemá žádný efekt na PHY.
*TX\_ER -- Pokud PHY vysílá jeden či více symbolů, která nejsou součástí dat nebo předčasně dojde k přerušení rámce, tak tento pin synchroně na základě hodinového signálu TX\_CLK a nastaveného signálu TX\_EN vyhodí chybu.
*RX\_DV -- Pin označující, že přijatá data jsou validní. Tento pin je řizen PHY a je synchroní s RX\_CLK.
*RXD -- Čtveřice datových pinů (RDX[3:0]) ovladané PHY, které slouží k přenosu dat z PHY do RS. RXD[0] přenáší nejméně významný bit.
*RX\_ER -- Pin řízený RHY, který indikuje RS, že nastal error detekovaný někde v průběhu přijímaní rámce.
Pokud je RX\_DV odpojen nemá žadný vliv na RS.
*CRS -- Tento pin je aktivní pokud příjemce nebo odesilatel je zaneprázdněn a neaktivní pokud jsou idle.
Není synchroní jak na TX\_EN tak ani na RX\_EN.
*COL -- Tento pin vyvolán PHY pokud na mediu nastane kolize.
*CRS a COL -- Piny jous neaktivní pokud ve stavu idle. Nejsou synchroní na TX\_CLK tak ani na RX\_CLK. Nedefinovane chovani COL a CRS pokud bit 0.8 je aktivni.
%Reprezentuje pokud josu data na odeslani nebo prijeti.
\enditems

Prenos dat je popsan formatem: \hfill \break
<inter-frame><preambule><začátek datového rámce><data><konec datového rámce>. \hfill \break
Pričemž byte je rozdělen na dvě poloviny tzv. nibble. A ty jsou odeslány ve dvouc částech po TXD. 
Jak mužeme vidět na obrázku X.
%popis  <inter-frame><preamble><sfd><data><efd>  ?

\medskip
\clabel[NIBBLE]{Rozdělení bytu po datových pinech}
%\picw=13cm \cinspic MIItoISO.png
\picw=15cm \cinspic images/802_nibble.pdf
\caption/f Obrázek popisuje jak je rozdělen byte, který je následně odeslán po TDX \cite[IEEE8023].
\medskip



\medskip
\clabel[transiotionMII]{Komunikace po MII bez kolize.}
%\picw=13cm \cinspic MIItoISO.png
\picw=15cm \cinspic images/802_miicom.pdf
\caption/f Obrázek ukazuje jak vypada komunikace po MII bez nastání kolize \cite[IEEE8023].
\medskip

\label[MDIOs]

\secc Menežovací rozhraní a příslušný registrový prostor

Menežovací čast sběrnice označena STA používá k nastavování a čtením hodnot mezi MAC a PHY.
Toto rozhraní je specifikovano dvou linkovym seriovym rozhraním.
Toto rozhraní obsahuje dva datové vodiče:
\begitems
*MDC -- Pin určující časovou základnu pro MDIO signal. MDC je aperiodický signál, který nemá žádnou minimální ani maximální frekvenci, avšak by se měl pohybovat v rozmezí od 160 ns do 400 ns.
*MDIO -- Pin, který je vstupně výstupní a posílají se po něm signály oboumi směry je proveden třístavovým obvodem mezi PHY a STA.
Používá se k přenosu kontrolních zpráv mezi těmito zařizeními. MDIO je synchroní s MDC a informace jsou řizeny z PHY.
\enditems

Formát menežovacích rámců je popsán tabulkou \ref[MDIOframe]. A data jsou odesílána po MDIO z levé strany tabulky do pravé.

\midinsert \clabel[MDIOframe]{Formát menežovacích rámců.}
\ctable{lllllllll}{
 \hfil   operace  & PRE & ST & OP & PHYAD & REGAD & TA & DATA & IDLE \crl \tskip4pt
          cteni   & 1...1 & 01 & 10 & AAAAA & RRRRR & Z0 & D...D & Z \cr
          zapis   & 1...1 & 01 & 01 & AAAAA & RRRRR & 10 & D...D & Z \cr
}
\caption/t Format menezovacich zprav odesilany z leva do prava \cite[IEEE8023] (Tabulka 22-12).
\endinsert

V tabulce \ref[MDIOframe] mužeme vidět hodnotu preambule (PRE), která je posloupnost 32 bitu hodnoty logické 1 sloužící k synchronizaci s MDC.
Začátek rámce označný ST je 01.
Poté následuje OPerační kód (OP), který určuje druh operace, čtení nebo zápis.
PHYAD je fyzická adresa zařízení o délce 5 bitu tj. až 32 zařízení, které můžeme naadresovat.
První bit adresy je nejvíce významný.
Další položkou je adresa registru (REGAD) číslovaná jako níže uvedene registry. Dalším blokem je TA.
Při čteni hodnota vysoké impedance slouží k prohození odesilatele, kdy se aktivní odesilatel přehodí do stavu vysoké impedance a nechá vysílat PHY, tak jak můžeme vidět na obrázku \ref[mdioread].
Pri zápisu zde není místo, na prohození odesílatelů a odesílá se logická 1.
Data jsou 16 bitová hodnota, jako je velikost registrů.

\medskip
\clabel[mdioread]{Čtení registru po STA.}
%\picw=13cm \cinspic MIItoISO.png
\picw=15cm \cinspic images/802_miimdio.pdf
\caption/f Obrázek ukazuje čtení z registru po komunikaci po seriové lince STA \cite[IEEE8023].
\medskip

Za pomoci STA, MAC může vyčítat z registrového prostoru PHY. Část registrového prostoru je definována tež standardem 802.3.
Registrový prostor obsahuje dva základní sety registrů kontrolní a stavový.
Všechny PHY, které podporují MII by měli používat tuto sadu registrů.
Stavový a kontrolní registrové prostory můžeme vidět v příloza XXX.
Tyto registry specifikují základní vlastnosti pro 100Mb/s a 1Gb/s PHYs.
Registry 2-14 jsou částí rozšířeného regisrového prostoru \cite[IEEE8023] (22.2.4).


% Ne do DP
%\sec GMII

%The Gigabit Media Independent Interface (GMII) is similar to the MII. The GMII uses the MII management interface and register set specified in 22.2.4. These common elements of operation allow Station Management to determine PHY capabilities for any supported speed of operation and configure the station based on those capabilities. In a station supporting both MII and GMII operation, configuration of the station would include enabling either the MII or GMII operation as appropriate for the data rate of the selected PHY. Most of the MII and GMII signals use the same names, but the width of the RXD and TXD data bundles and the semantics of the associated control signals differ between MII and GMII operation. The GMII transmit path clocking also differs significantly from MII clocking. MII operation of these signals and clocks is specified within Clause 22 and GMII operation is specified within Clause 35.
%GMII obsahuje 3ti zakladni registr Extended status reg(15)

\secc Reduce Media Independet Interface


Reduce Media Independet Interface (RMII) je obdoba MII, která má snížený počet pinů z 16 na 8 \cite[RMII].
RMII je nízkonákladova obdoba MII, která přidává pouze reconciliation layer (RS) na oboch stranách MII či náhradu aktualního RS v MII \cite[RMIIcon].
RMII používá stejné menežovací rozhraní jako MII definované IEEE 802.3u \cite[IEEE8023] popsané \ref[MDIOs].
Datové vodiče RMII jsou pouze 2 bity široké.
Dále je jsou sjednoceny signály COL a CRS do jednoho signálu CRS\_DV.
Detekce kolizí na MAC je logický součin signálů TX\_EN a CRS\_DV.
Zdrojem hodinového signálu je vrstva MAC nebo je generován externími hodinami.

%TODO pridat obrazek RSl

% TODO
%\sec linux

%\sec info kernel

%\sec loading proces

%\sec disribuce

%\secc openwrt

\sec Subsystém operačního systému Linux

Tato kapitola obsahuje informace o tom, co to je ovladač. A jak je ovladač přepínače \ref[DSA] a usb zařízeních integrovaný \ref[USBst]  v operačním systému Linux.

\secc Co je to ovladač

Ovladač je kus programu (algoritmu), který zpravuje nebo kontroluje určité zařízení připojené k počítači.
Ovladače vytvářejí softwarové rozhraní pro hardwarové zařízení, které umožňují operačnímu systému ovládat tento hardware aniž by uživatel věděl jak přesně funguje \ref[LDD3].

\secc device

%{\em } 
%\fnote{\url{Odkaz na IEEE}}}.

Každé zařízní na té nejnižší úrovni je v operačním systému Linux je reprezentováno instancí struktory {\em struct device}.
Struktora device obsahuje informace o tom co zařízení potřebuje aby správně fungoval v systému.
Mnoho subsystému uchovává informace o této struktuře, na které stavějí.
Je velmi optížné dnes narazit na zařízení, které je reprezentováno pouze struktorou {\em struct device}.
Místo toho struktury obecně implementují vyšší reprezentaci zařízení \ref[DEV].

\secc miibus

\label[DSA]

\secc Distributed Switch Architecture

Distributed Switch Architecture (DSA) je protokol pro menežování hardwarových přepínačů.
Obsahuje MII menežovací registry, příkazy pro nastavení přepínače a formát ethernetových hlaviček, který signalizuje na kterém portu byl paket přijat nebo ktery se hodlá odeslat \ref[DSA].

Tento driver podporuje přepínače, které jsou připojeny způsobem zobrazeným na obrázku \ref[DSAcon].

%+-----------+       +-----------+ \hfill \break
%|           | RGMII |           |
%|           +-------+           +------ 1000baseT MDI ("WAN")
%|           |       |  6-port   +------ 1000baseT MDI ("LAN1")
%|    CPU    |       |  ethernet +------ 1000baseT MDI ("LAN2")
%|           |MIImgmt|  switch   +------ 1000baseT MDI ("LAN3")
%|           +-------+  w/5 PHYs +------ 1000baseT MDI ("LAN4")
%|           |       |           |
%+-----------+       +-----------+

Tento ovladač přepínače reprezentuje každý port jako oddělené síťove rozhraní.
Pollování, stav portů, MII menežment rozhraní je proveden za základě rozhraní ethtool \fnote{Copak je ethTool XXX}.


DSA podporuje i propojení mezi switchy, které pak mužem take ovladat.

%        +-----+          +--------+       +--------+
%        |     |eth0    10| switch |9    10| switch |
%        | CPU +----------+        +-------+        |
%        |     |          | chip 0 |       | chip 1 |
%        +-----+          +---++---+       +---++---+
%                             ||               ||
%                             ||               ||
%                             ||1000baseT      ||1000baseT
%                             ||ports 1-8      ||ports 9-16




\label[USBst]

\secc usb driver

Linuxové jádro podporuje dva druhy USB zařízení ovladače na hostovském systému a ovladače na zařízení \cite[LDD3].
USB ovladač na hostovském systému, jak může být podle názvu jasné, běží systému ke kterému je USB zařizení připojeno.
Ovladače na zařízení neboli \"USB gadget drivers\"  zařízení připojojené k hostovskému systému vypadá jako hostovská stanice připojena přez USB.
Na obrázku níže mužeme vidět USB stack, kde USB muže existovat v několika různých subsystémech (net, block, char ...). USB core dává rozhraní pro USB ovladače, které chtějí kontrolovat a přistupovat k hardwaru.


Koncové body USB jsou popsány v jádře Linuxu strukturou {\em struct usb\_host\_endpoint}.
Tato struktora obsahuje informace o realném koncovém bodu v struktuře {\em struct usb\_endpoint\_descriptor}.
Struktora {\em struct usb\_endpoint\_descriptor} popisuje data pomocí příjmutých dekriptorů.

USB rozhraní je složeno z několik koncových bodů, které tvoří jedno logické připojení jako je napřiklad myš, klávesnice, video atd. Některá USB zařízení mohou mít více rozhraní.
USB repoduktory mohou mít dvě rozhraní jako je USB klávesnice pro tlačítka a USB audio stream \ref[LDD3].%TODO presunout nahoru
Pro každé rozhraní Linuxové jádro používá jeden hardwarový ovladač.
Toto rozhraní je popsáno strukturou {\em struct usb\_interface}.
Tato struktura je to co USB core předává USB ovladačům.
Každé zařízení je vázáno konfigurací a jedno USB zařízení muže mít těchto konfigurací více.
Tyto konfigurace se mouhou v průběhu běhu zařízení měnit.
A však Linuxové jádro nemůže obsluhovat více konfigurací v jeden čas. Linuxové jádro popisuje konfigurace v struktuře {\em struct usb\_host\_config} a celé USB zařízení strukturou {\em struct usb\_device}.
Ovladače USB zařízení  obyčejně přepisují data z {\em usb\_interaface} do {\em struct usb\_device}.

Samotné struktury {\em struct usb\_device} a {\em struct usb\_interface} jsou zobrazeny v sysfs\ref[SYSFS] jako jednotlivá soubory.

Na řádce níže můžeme vidět ukázkovou cestu k {\em struct usb\_device} reprezentovanou pomocí sysfs:

\begtt  /sys/devices/pci0000:00/0000:00:09.0/usb2/2-1  \endtt

K výše uvedené cestě k USB zařízení je na řádku níže zvírazněno USB rozhraní reprezentováno strukturou {\em struct usb\_interface} je pojmenovan podle formátu root\_hub-hub\_port:config.interface (pro hlubší stromy je schéma root\_hub-hub\_port-hub\_port:config.interface):

\begtt /sys/devices/pci0000:00/0000:00:09.0/usb2/2-1/2-1:1.0 \endtt

Pro komunikaci se všemi USB zařízeními Linuxové jádro používá USB request block (urb). Urb se používá k posílaní a příjmu dat pro specifický koncový bod USB a specifické USB zařízení asynchroním způsobem.

%\secc net\_device XXX

\label[USBNET]

\secc usbnet

Usbnet je subsystem Linuxového jadra, který umožňuje ovládat USB-síťové zařízení jako je ethernet, DSL, IDSN atd \ref[USBNET].
Usbnet je to genericky USB síťový framework, který pracuje na různých rychlostech a nad různými protokoly.


\secc System Filesystem

SYstem FileSystem (Sysfs) je charakteristika Linuxového jádra od verze 2.6, která umožňuje kernelovému kódu exportovat informace do userspace za použití paměti VFS\ref[SYSFS].
Hlavním učelem je reprezentovat objekty, jejich atributy a jejich vstahy navzájem.
Většina atributů je reprezentována souborem jež je ve formátu ASCII a obsahuje pouze jednu hodnotu.
Sysfs poskytuje dvě složky:

\begitems
*kernel programing interface -- Slouží k exportování a importování položek (viz níže) skrz sysfs do kernelu
*user interface -- Slouží k vidění a manipulaci těchto položek (viz níže), které mapuje zpět na objekty v kernelu
\enditems

Mapování mezi objekty můžeme vidět v \ref[SYSFSmap].

\midinsert \clabel[SYSFSmap]{Mapováni objektů a atributu v sysfs.}
\ctable{ll}{
 \hfil  Interni   & Externi  \crl \tskip4pt
        Kernelovské objekty    & Složky  \cr
        Atributy objektů     & Soubory  \cr
        Vazby mezi objekty   & Symbolické odkazy  \cr
}
\caption/t  Tabulka prevzana z \cite[SYSFS].
\endinsert

V systemu můžeme vidět tyto zařízení vidět ve složce  {\em /sys/ }.
Složka /sys/ může vypadat následovně:

%/sys/
%|-- block
%|-- bus
%|-- class
%|-- devices
%|-- firmware
%|-- module
%‘-- power


\chap Použitý hardware

\label[T20]

\sec Colibri T20

Colibri T20 je počítačový modul postaven na základě NVIDIA Tegra 2 embeded systému na čipu (SOC) \cite[SOC].
SOC modul Tegra 2 je založen na dvoujádrovém procesoru Cortex A9 se symetrickým procesorovým jádrem od firmy ARM s rychlostí okolo 1 Ghz.
Jelikož se jedná o SOC obsahuje tento čip také mnoho bluků jako je například Audio/Video rekordér.
Grafický čip s podporou 2D rendrovaní a 3D pixel a vektor shadrem.

Čip obsahuje periferie popsané na blokovém diagramu na obrázku \ref[SOC].

\medskip
\clabel[SOC]{Colibri T20 blokový diagram.}
\picw=15cm \cinspic images/soc.pdf
\caption/f Obrázek popisuje části integrované na modulu Colibri T20 \cite[SOC].
\medskip

Toradex dodává podporu BSP pro Linuxové jádro 3.1.

%chtelo by to swap Colibri a ARM ci spojit...
\sec NVIDIA Tegra 2

Dvoujadrový procesor ARM Cortex A9 symmetric MPCore je jeden z aplikačních procesorů vyvíjený firmou ARM.
Tento ARM je 32-bitová RISC architektura vyvíjená společností ARM Holding věnující se vývoji procesorových jader.
Tento procesor porporuje out-of-order a spekulativní provádění instrukcí.
Má plnou podporu koherence pamětí pro symetrické procesory.
Toto jádro dosahuje vysokého grafického výkonu, ale také výpočetního výkonu viz. \cite[BENCH].
Procesor též obsahuje jednotku pro výpočty s plovoucí řádovou čárkou.

Popis jádra NVIDIA Tegra 2 můžeme vidět na obrázky níže \ref[TEGRA].

\medskip
\clabel[TEGRA]{Popis jádra NVIDIA Tegra 2.}
\picw=15cm \cinspic images/tegra2.pdf
\caption/f Obrázek popisuje vnitřek proceosru NVIDIA Tegra 2 \cite[TEGRA].
\medskip

\label[ASIX]

\sec Asix AX88772b


AX88772b je zákaznický integrovaný obvod, který umožňuje plug-and-play Fast Ethernet internetové připojení pro zařizení používající stantardizované USB \ref[USB].

Jako jediný z výrobních řad dosahuje revize AX88772bli teplotní rozsah od -40°C až +85°C.

AX88772b má modifikovatelnou vícefunkční sběrnici, která umožňuje připojit RMII \ref[RMII] nebo jako Reverse-RMII pro MAC-to-MAC připojení k nějakému mikrokontroléru s ethernet MAC RMII rozhraním.

MAC rozhraní je plně kompatibilní se standardy IEEE~802.3, IEEE~802.3u.
%neudelat 802.3 chap?

\medskip
\clabel[ASIX]{.}
\picw=15cm \cinspic images/asix.pdf
\caption/f Obrázek popisuje  \cite[ASIX].
\medskip
%jake umoznuje propojeni po rmii?

\label[MARVELL]

\sec Marvell 88E6065

Marvell 88E6065 je 5+1 portový ethernetový přepínač integrovaný na jednom čipu.
Obsahuje  pět portů na fyzicke vrstvě 10~BASE-T/100~BASE-TX, znichž dva porty můžou být použity pro optickou linku 100~BASE-FX.
Šestým portem je nezávyslí Fast ethernet MAC.
Zařízení má high-speed, non-blocking čtyřúrovňový QoS.
PHY podporuje plag-n-play s možností Auto-Crossoer, Auto-polarity a Auto-nogotiation.
Zařízení dále odporuje 64 z 4096 802.1Q WLAN s 3 úrovňovou ochranou.
Má dva RMII/MII/SNI rozhraní, která mohou být připojena k menežovacímu rozhraní nebo ke kontroleru s integrovanou MAC.

MAC a PHY rozhraní jsou plně kompatibilní se standardy IEEE~802.3, IEEE~802.3u a IEEE~802.3x.
Zařízení může být konfigurováno přez STA nebo může nastavení načítat z EEPROM paměti.

Výjmečně se může Marvell 88E6065 využívat podpory pro směrovače a výchozí brány.
%jake umoznuje porjojeni RMII

\label[SWITCH]

\sec Menežovatelný přepínač


Menežovatelný přepínače umožňuje konfiguraci ethernetoveho přepínače za pomoci USB kontroleru Asix AX88772b od firmy Asix \ref[ASIX], ke kterému je připojena externí pamět eeprom M93-C66.
Tento kontrolér je připojen RMII rozhraním k čipu Marvell 88E6065 \ref[MARVELL].
Marvell 88E6065 je ethernetový přepínač, který má k sobě kromě zmíněného RMII portu připojeno pět ethernetových portů a taktéž je kněmu připojena externí pamět eeprom.
Oba tyto čipy mají též vlastní krystal na frekvenci 25~Mhz.
Deska je taktéž osazena regulátorem napětí.

Desku menežovatelného přepínače je vyrobena firmou Retia a.s. \fnote{Firma zabívající se elektronickými vojeskými systémy, zaznomovými zařízeními a lokalizačních a bezpočnostních systémů. \url{www.retia.cz}}} na základě jejich návrhu a zapojení popsaných v přílohách \ref[zapojeni] \ref[zapojeni2] \ref[zapojeni3].

\chap Implementace

Tato kapitola rozebírá update systému pro modul Colibri T20 \ref[T20], který byl zapotřebí pro další implementaci ovladačů pro aktuální Linuxové jádro.
Dále se zabívá způsobem oživení desky \ref[SWITCH] a následným popisem tvorby ovladače pro menežovatelný přepínač.

Vývoj probíhal na vyvojovém kitu s modulem Colibri T20.
Pro vývoj ovladače bylo zvoleno jádro z upstreamové větve kernelu
\fnote{\url{http://git.kernel.org/cgit/linux/kernel/git/torvalds/linux.git}}}
konkrétně se začlo s verzí 3.17 a došlo až k přeportování ovladače na verzi 3.19.
Portace upstreamového jádra je popsána v \ref[UPSTREAM].

\label[UPSTREAM]

\sec Portace upstreamové linuxové jádro a zavaděče

Portace upstramového jádra vyžaduje několik nástrojů a programů.

\sec U-Boot

V první řadě je potřeba naistalovat zavaděč, který se postará o zavedení Linuxové jádra.
Pro tento úkol byl zvolen U-Boot, který používá již výrobce desky Colibri T20.
Jen byla zvolena jeho novejší verze
\fnote{U-Boot je možný stáhnout z repozitáře \url{git://git.denx.de/u-boot.git}}}.

Potup nahrání nového U-Bootu je popsán v příloze \ref[UPUBOOT] jež byl inspirován \ref[UPLINUXT20].

\sec Linuxové jádro



\sec navrh

\sec Rozchozeni hardwaru

Rozchozeni hardwaru byla jedna z nejobtiznejsich casti.

Pricemz se zacalo s testovanim USB kontroleru Asix AX88772b. A po rozchozeni komunikacnich interfacu bylo pokracovano s ladenim ethernetoveho prepinace 88E6065.
Posledni casti bylo testovani ethernetovych portu.

Pricemz bylo objeveno nekolik chyb v navrhu zarizeni. Ktera na zaklade mojich pripominek byla opravena(XXX to zaveru).

\secc asixu

Kdy byla objevena zavada na pripojeni kontroleru.
Tato zavada se projevovala chybnou komunikaci na USB sbernici. Tu to chybnou komunikaci muzete videt na obrazku \ref[usbbuserr].

\medskip
\clabel[usbbuserr]{Chybna komunikace na usb sbernici.}
%\picw=13cm \cinspic MIItoISO.png
\picw=6cm \cinspic a.png
\caption/f Na obrazku muzeme videt chybnou komunikaci na usb sbernici mezi PC a usb kontrolerem Asix AX88772b zavinou spatnym nepripojenim blokovacich kondenzatoru.
\medskip

Po uprave zapojeni kontroleru. Zacal USB kontroler odpovidat na dotazy.

Dale bylo potreba nastavit vhodne chovani cipu Asix aby zacalo fungovat manazovaci rozhrani RMII sbernice. Bylo zvoleno zapojeni, ktere aktivuje RMII. Pote, jiz bylo mozne vycitat data ze zarizeni Marvell.

\secc marvell

Na zarezeni 88E6065 bylo provedeno nekolik zasahu. Byli pridany pull-up a pull-down rezistory pro spravne a deterministicke chovani prepinace.
Dale byl doplne kondenzator, ktery zarizeni vyresotoval po dobu konstanty RC po pripojeni pripinace k napajeni. Tento kro byl dulezity ponevadz vsechny zarizeni marvell chce vstupy pri startu v definovene hodnote. A stavalo ze pri zapnuti prepinace se hodnoty ruznych zapnuti ruzne menili.

\sec zpracovani

\sec celek?

\sec odezva komunity

Odezva komunity byla dobra - po odeslani RFC patche (*pozn pod carou) se mi dostalo odpovedi behem pouhych par hodin po jeho odesalni.

\chap Otestujte implementované řešení zejména s ohledem na propustnost sítě a stabilitu

Tato kapitola se zabiva

\sec propustnost

\sec porovnani s jinym eth interfacem

\sec konfigurace 

co se testuje a jak se testuje(co pisu za prikazy) -> vysledky

\sec logterm

\chap Zaver

funguje, odezva, linuxova komunita, dosazene vysledky, vykon
MII,RMII,GMII,RGMII,SGMII,QGMII,XAUI
