\chap Uvod

aaaaa

\chap usb

\label[usb]

Universal Serial Bus (USB) je standart vyvyjeny od roku 1994 pro pripojeni periferii k pocitaci.
USB melo nahradit seriove a paralerni porty a vsak v soucasne dobe nahrazuje take PS/2, audio, video a take umoznuje prenaset ethernetove ramce.
USB je asymetricka topoligii pripojena do hvezdicove topologie - konkretne stromu. Pricemz USB huby vytvareji jednotlive uzly tohoto stromu a zarizeni tvori jejich listy. Maximalni hloubka toho stromu je 5 a k usb muze mit maximalne 127 zarizeni.
USB komunikace je zalozena na logickych kanalech tzv. pipes. Kazdemu vystupnimu kanalu by mel odpovidat jeden vstupni.
USB muze mit maximalne 30 endpointu. Tyto endpointy jsou inicializovany v prubehu enumarace USB zarizeni, ktere probiha po kontrolnim kanalu 0, ktery maji vsechna zarizeni.
Krome kontrolniho prenosu existuje izochroni, prerusovany a blokovy prenos.

\begitems
*Kontrolni - Slouzi ke konfiguraci zarizeni. Kazde zarizeni disponuje timto endpointem. Tento endpoint slouzi k enumeraci
*Izochroni - Slouzi  ke stalemu prenosu vetsich dat. Je garantovana latence, ale neni garantovao doruceni. Velmi vhodne pro audio a video.
*Prerusovany - urceny pro caste prenosy maleho mnozstvi dat. Garantovano doruceni a sircka pasma. Je defonovano maximalni zpozdeni
*Blokovy - je urcen pro prenos velkeho mnozstvi dat, ale nezarucuje pasmo ci latenci(prenos dat) viz prednaska Novaka
\enditems

Enumerace je posloupnost standardizovanych prikazu, ktery zapocal host. V prubehu enumerace se predavani tzv. deskriptory. Ktere obsahuji tyto dulezite informace o zarizeni. Na zaklade informace v deskriptorech se na zarizeni pouzije specialni ovladac.
Tento standart se v soucasne dobe rozvinul u umoznuje pripojit temer jakekoli zarizeni. Jako jsou ukazovaci zarizeni, pametova zarizeni, tiskarny, komunikacni zarizeni a mnohe dalsi.
Tato zarizeni jsou roztrizena do trid z dovodu s jednoceni druhu chovani jako jsou HID, PID Printer, Mass Storage a mnohe dalsi.
Toto sjednocene chovani ma za dusledek spolecnych ovladacu, ktera tato zarizeni mohou vyuzivat. 
A vsak nektera z techto zarizeni se do techto trid nevnestnaji ci maji rozsirene chovani a potrebuji specailni pristup.

\sec mii rmii

\secc popis

\secc komunikace

\secc regspace

\secc linux?

\sec colibri t20

\secc arm arch



\secc perif

\sec asix

\secc popis

\sec marvell

\secc popis

\sec linux

\secc Co to je Ovladac?

\secc info kernel

\secc loading proces

\sec disribuce

\secc openwrt

\sec linux sub

\secc miibus

\secc device

\secc dsa

\secc usbnet

\secc sysfs

\sec Prakticka cast

\secc vytvor

\secc hw

\secc sw

\secc testing

\secc results

\secc Otestujte implementované řešení zejména s ohledem na propustnost sítě a stabilitu

\chap Zaver


\chap Zdroje:

http://www.usb.org/developers/docs/
MII,RMII,GMII,RGMII,SGMII,QGMII,XAUI
