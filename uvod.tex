\chap Úvod
%TODO list
% usb obr, enumaration, mazna edpoint+interface
% ASIX zapojeni + asix zapojeni RMII
Tu se deje kupa zajimavych veci ....

V kapitole pouzitech technologie \ref[pouztech] se dozvime teorii o XXX.


\sec Aktualni stav problematiky


%\chap Navrh

\label[pouztech]

\chap Použité technologie

Tato kapitola je rozebírána použitá technologie.
V první části je rozebrána sběrnice USB.
Další významnou sekcí je rozhraní Media Independet Interface, který předpokládá znalost modelu OSI, který je tu též uveden.
Ke konci této kapitoly bylo rozepsáno jak vypadájí ovladače v linuxu a tvář linuxového subsystému Distributed Switch Architecture a k němu přilehlé subsystémy.

\label[USB]

\sec Univerzální Sériová Sběrnice

Univerzální Sériová Sběrnice je technický standard vyvíjený od roku 1994 pro připojení periferií k počítači.
Tato sběrnice měla nahradit pomalé sběrnice (jako jsou seriová linka, paralelní port, PS2 a mnohé další) jednou sběrnicí \cite[LDD3].
V dnešní době USB umožňuje přenášet velké množství dat s rychlostí až 640 MB/s u USB zařízení verze 3.0.
Díky tomu se použití USB rozšířilo a nyní podporuje skoro každý druh zařízení, která umožňují přenašet video, audio a dokonce i ethernetové rámce a pakety.
Podporovaná zařízení jsou rozdělena do tříd z důvodu sjednocení funkcionality a chovaní zařízení.
Nejznámějšími třídami jsou HID, Printer a Mass Storage, které jsou více popsány ve vývojářské dokumentaci \ref[USB].
Toto sjednocené chování má za důsledek společných ovladačů, která tato zařízení mohou využívat.
Problém s některými druhy zařízení je takový, že je nejde do daných tříd zařadit, popřípadě mají rozšířené chování a potřebují speciální přístup.

\secc Rysy USB zařízení

USB je deterministická sběrnice (Master/Slave), která podporuje detekci připojeného zařízení za běhu (plug\&play) a následnou automatickou konfiguraci (hotswap).
USB síť se zkládá ze třech druhů zařízení \cite[USB]:
\begitems
*Hostitel -- je zařízení typu Master a je jako jediný v systému, řídí komunikaci a obvykle integruje rozbočovač označen jako kořenový.
Přiděluje zařízením unikátní adresu v síti.
*Rozbočovač -- distribuuje datové toky a identifikuje připojení a odpojení dalších zařízení.
*Zařízení -- Pojmem zařízení je míňeno koncové zařízení s požadovanou funkcionalitou, které odpovídá na kontrolní zprávy.
\enditems

USB je asymetrická sběrnice s jedním zařízením typu Master, která spíše vypadá jako strom složený z linek bod-bod.
Přičemž USB huby vytvářejí jednotlivé uzly tohoto stromu a zařízení tvoří jejich listy.
Linky USB sběrnice jsou 4 vodičové obsahující diferenciální pár datových linek, napěťový a zemnící vodič, které připojují zařízení a USB huby.
USB umožňuje připojit až 127 zařízení v rámci jedné USB sítě o hloubce maximálně pěti rozbočovaču.


USB je asymetrická sběrnice, kterou lze popsat jako strom složený z linek bod-bod.
USB huby vytvářejí jednotlivé uzly tohoto stromu a zařízení tvoří jejich listy.
USB umožňuje připojit až 127 zařízení v rámci jedné USB sítě o hloubce maximálně pěti rozbočovačů, tak jak můžeme vidět na obrázku \ref[USBhub].

\medskip
\clabel[USBhub]{USB tier.}
\picw=13cm \cinspic images/technet.microsoft.com.jpg
\caption/f Obrázek popisuje jak lze USB zapojit \cite[MSTECH].
\medskip

Linky USB sběrnice jsou 4 vodičové vedení obsahující diferenciální pár datových linek, napěťový a zemnící vodič, která připojují zařízení a USB huby.

Díky tomu USB umožňuje jednoduchý mechanismus detekce a enumerace, neboli průběhu konfigurace (více informací níže \ref[USBkom]), za běhu systému, který je řízen a konfigurován automaticky hostitelem.


\label[USBkom]

\secc Komunikace

USB komunikace je založena na logických kanálech tzv. rourách. Každému výstupnímu kanálu by měl odpovídat právě jeden vstupní.
Koncový bod obsahuje rouru s definovaným směrem.

USB muže mít maximalně 30 koncových bodů. Koncový bod je roura s definovaným směrem.
Tyto koncové body jsou inicializovány v průběhu enumarace USB zařízení, které probíhá po kontrolní rouře označené číslem 0, kterou mají všechna zařízení společnou.
Každý koncový bod obsahuje rouru s předem definovaným typem přenosu:

\begitems
*Řídící -- Obousměrná roura sloužící ke konfiguraci zařízení. Každé zařízení disponuje tímto druhem roury. Má rezervovanou určitou přenosovou kapacitu.
*Izochronní - Jednosměrná roura sloužící ke stálému přenosu většího objemu dat. Má garantovanou latenci, přenos však není spolehlivý. Tento typ roury je vhodný pro audio a video.
*Přerušovací - Jednosměrná roura sloužící pro časté přenosy malého množství dat. Má garantovanou šířku pásma. A přenos dat je spolehlivý. V případě chyby se přenos opakuje.
*Blokový - Jednosměrná roura sloužící k přenosu velkého množství dat. Nemá rezervovanou žádnou přenosovou kapacitu ani dobu odezvy. Komunikace po této rouře je spolehlivá a v případě chyby se přenos opakuje.
\enditems

Enumerace je posloupnost standardizovaných příkazů, kterou započal hostitel.
V průběhu enumerace se předávají deskriptory, které obsahují důležité informace o zařízení.
Hierarchiji deskriptorů můžeme vidět na obrázku \ref[USBhier].

\medskip
\clabel[USBhier]{Hierarchije USB deskriptorů.}
\picw=13cm \cinspic images/IMG019.pdf
\caption/f Obrázek popisuje jak lze USB zapojit \cite[LUSB].
\medskip


\begitems
*Dekriptor zařízení -- Obsahuje informace jako je ID vyrobce a produktu.
Též obsahuje informace o třídě a maximalní delku paketů, které může roura 0 přijmout.
*Konfigurační deskriptor -- Oznamují jak mají být napájeni, maximální proud, počet rozhraní. Těchto dekriptorů může být více.
*Deskriptor rozhraní -- Tento dekriptor informuje o funkcionalitě celého celku jako je třídá koncových zařízení a počet koncových bodů.
*Deskriptor koncového bodu --Používá se k přesnémuj popisu koncového bodu.
*Textový dekriptor -- Tento dekriptor přenáší pouze čitelné informace o zařízeních kódované v Unicodu
\enditems

Na základě informací v deskriptorech se operační systém rozhoduje, který druh ovladače přiřádí danému druhu zařízení.

%LDD3 pekny popis zarizeni http://www.makelinux.net/ldd3/?u=chp-13-sect-4
%MAC podporuje 802.3 a 802.3u MAC funkce, jako je prijem a odeslani ramce, kontrolu CRC, duplex mode, forwarding, flow-control, detekce kolizi atd.

\medskip \medskip \medskip \medskip

\label[MII]

\sec Media Independent Interface

Media Independet Interface je typ rozhraní, který umožňuje připojení nezávyslého na procesoru a k fyzickém médiu.
Media Independent Interface (MII) je specifikováno standardem IEEE 802.3 v kapitole 22 \ref[IEEE8023].
MII propojuje dvě vrstvy ISO/OSI a to vrstvu Data Link, konkretne jeji cast Medai Access Constroll (MAC) a vrstvou fyzickou (PHY).


%popis MAC a PHY???

\medskip
\clabel[MIItoISO]{Spojitost mezi MII a OSI/OSI}
\picw=16cm \cinspic images/802_iso.pdf
\caption/f Obrázek popisuje spojistost mezi rozhranim MII, modelem OSO/OSI a modelem IEEE 802.3 CSMA/CD LAN \cite[IEEE8023].
\medskip

Na obrázku \ref[MIItoISO] můžeme vidět připojení MII na Reconsiliation Sublayer (RS) a PCS/PLS.
RS je podvrstva, která mapuje signály z MII na MAC/PLS obsluhu.
Mapování těchto signálů můžeme vidět na obrázku \ref[RStoMII].
PLS/PLC jsou vrstvy, které se starají o kódování a dekódování signálů.


%asi zbytecny detaili
\medskip
\clabel[RStoMII]{Mapování signálů RS to MII}
%\picw=13cm \cinspic MIItoISO.png
\picw=15cm \cinspic images/802_rs.pdf
\caption/f Obrázek popisuje mapování signalů mezi RS vstupem a výstupem a také STA na MII \cite[IEEE8023].
\medskip

Až 32 PHY jednotek může být obsluhováno jedním menežovacim MII rozhraním.
MII podporuje dva datové toky a to je 10~Mb/s a 100~Mb/s.
Funkcionalita je identicka na obou datových tokách a liší se pouze v nominální frekvenci hodin.

Fyzické rozhraní MII se skládá ze dvou druhů sběrnic a to datové \ref[MIIdat] a menežovací \ref[MIImdio].

\label[MIIdat]

\secc Datová část MII rozhraní

Datová část MII disponuje několika druhy signálu, které mužeme vidět na pravé straně obrázku \ref[RStoMII] popsané níže.

\begitems
*TX\_CLK -- Pin vysílající refeneční hodinový signál pro synchronizaci pinů TX\_EN, TXD a TX\_ER, která vede z RS do PHY. 
Zdrojem TX\_CLK je PHY. 
Frekvence tohoto signálu by měl být 20\% z nominální hodnoty přenosu dat +- 100ppm.
*RX\_CLK -- Pin vysílající refeneční hodinový signál pro synchronizaci pinů RX\_DV, RXD a RX\_ER, která vede z PHY do RS.
Zdrojem RX\_CLK je PHY. 
RX\_CLK může mít referenční hodnotu z přijímaných dat nebo muže být odvozena z nominální hodnoty tak jako u TX\_CLK.
*TX\_EN -- Pin indikující, že RS je připravena odesílat data.
*TDX -- Čtveřice datových pinů (TDX[3:0]) ovládána RS, která přenáší data synchroně na základě TX\_CLK. 
TDX[0] je nejméně významný bit.
Pokud je TX\_EN fyzicky odpojen TDX nemá žádný efekt na PHY.
*TX\_ER -- Pokud PHY vysílá jeden či více symbolů, která nejsou součástí dat nebo předčasně dojde k přerušení rámce, tak tento pin synchroně na základě hodinového signálu TX\_CLK a nastaveného signálu TX\_EN vyhodí chybu.
*RX\_DV -- Pin označující, že přijatá data jsou validní. Tento pin je řizen PHY a je synchroní s RX\_CLK.
*RXD -- Čtveřice datových pinů (RDX[3:0]) ovladané PHY, které slouží k přenosu dat z PHY do RS. RXD[0] přenáší nejméně významný bit.
*RX\_ER -- Pin řízený RHY, který indikuje RS, že nastal error detekovaný někde v průběhu přijímaní rámce.
Pokud je RX\_DV odpojen nemá žadný vliv na RS.
*CRS -- Tento pin je aktivní pokud příjemce nebo odesilatel je zaneprázdněn a neaktivní pokud jsou idle.
Není synchroní jak na TX\_EN tak ani na RX\_EN.
*COL -- Tento pin vyvolán PHY pokud na mediu nastane kolize.
*CRS a COL -- Piny jous neaktivní pokud ve stavu idle. Nejsou synchroní na TX\_CLK tak ani na RX\_CLK. Nedefinované chování COL a CRS pokud bit 0.8 je aktivní.
%Reprezentuje pokud josu data na odeslani nebo prijeti.
\enditems

Na obrázku \ref[transiotionMII] můžeme vidět jak může vypadat komunikace po MII rozhraní. bez chyby.

\medskip
\clabel[transiotionMII]{Komunikace po MII bez kolize.}
%\picw=13cm \cinspic MIItoISO.png
\picw=15cm \cinspic images/802_miicom.pdf
\caption/f Obrázek ukazuje jak vypada komunikace po MII bez nastání kolize \cite[IEEE8023].
\medskip

Rámec data, který je přenášen po MII rozhraní je doplněn o hlaviky a patičky popsáné  formátem:

\begtt
<vstupní rámec><preambule><sdf><data><edf>
\endtt

Kde jednotlivé položky mají následující význam:

\begitems
*vstupní rámec -- Perioda mezi rámci při která není specifikovaná, během kterého nesmí dojít k činosti na MII.
*preambule -- Definovaná posloupnost vstupních dat, která se skládá ze 7 oktetů (bytů).
Pravidelným střídánám jedniček a nul.
*sdf -- Oktet, který oznamuje začátek rámce.
*data -- Přenesená data.
*efd -- Konec rámce, který se provede schození signálu TX\_EN.
\enditems

Rámce jsou přenášeny tak, že oktet je rozdělen na dvě poloviny tzv. nibble. A ty jsou odeslány ve dvouc částech po datových linkách. Například po pinech TDX jak můžeme vidět na obrázku níže \ref[NIBBLE].

\medskip
\clabel[NIBBLE]{Rozdělení bytu po datových pinech}
%\picw=13cm \cinspic MIItoISO.png
\picw=15cm \cinspic images/802_nibble.pdf
\caption/f Obrázek popisuje jak je rozdělen byte, který je následně odeslán po TDX \cite[IEEE8023].
\medskip




\label[MDIOs]

\secc Menežovací rozhraní a příslušný registrový prostor

Menežovací čast sběrnice označena STA používá k nastavování a čtením hodnot mezi MAC a PHY.
Toto rozhraní je specifikovano dvou linkovym seriovym rozhraním.
Toto rozhraní obsahuje dva datové vodiče:
\begitems
*MDC -- Pin určující časovou základnu pro MDIO signal. MDC je aperiodický signál, který nemá žádnou minimální ani maximální frekvenci, avšak by se měl pohybovat v rozmezí od 160 ns do 400 ns.
*MDIO -- Pin, který je vstupně výstupní a posílají se po něm signály oboumi směry je proveden třístavovým obvodem mezi PHY a STA.
Používá se k přenosu kontrolních zpráv mezi těmito zařizeními. MDIO je synchroní s MDC a informace jsou řizeny z PHY.
\enditems

Formát menežovacích rámců je popsán tabulkou \ref[MDIOframe]. A data jsou odesílána po MDIO z levé strany tabulky do pravé.

\midinsert \clabel[MDIOframe]{Formát menežovacích rámců.}
\ctable{lllllllll}{
 \hfil   operace  & PRE & ST & OP & PHYAD & REGAD & TA & DATA & IDLE \crl \tskip4pt
          cteni   & 1...1 & 01 & 10 & AAAAA & RRRRR & Z0 & D...D & Z \cr
          zapis   & 1...1 & 01 & 01 & AAAAA & RRRRR & 10 & D...D & Z \cr
}
\caption/t Format menezovacich zprav odesilany z leva do prava \cite[IEEE8023] (Tabulka 22-12).
\endinsert

V tabulce \ref[MDIOframe] mužeme vidět hodnotu preambule (PRE), která je posloupnost 32 bitu hodnoty logické 1 sloužící k synchronizaci s MDC.
Začátek rámce označný ST je 01.
Poté následuje OPerační kód (OP), který určuje druh operace, čtení nebo zápis.
PHYAD je fyzická adresa zařízení o délce 5 bitu tj. až 32 zařízení, které můžeme naadresovat.
První bit adresy je nejvíce významný.
Další položkou je adresa registru (REGAD) číslovaná jako níže uvedene registry. Dalším blokem je TA.
Při čteni hodnota vysoké impedance slouží k prohození odesilatele, kdy se aktivní odesilatel přehodí do stavu vysoké impedance a nechá vysílat PHY, tak jak můžeme vidět na obrázku \ref[mdioread].
Pri zápisu zde není místo, na prohození odesílatelů a odesílá se logická 1.
Data jsou 16 bitová hodnota, jako je velikost registrů.

\medskip
\clabel[mdioread]{Čtení registru po STA.}
%\picw=13cm \cinspic MIItoISO.png
\picw=15cm \cinspic images/802_miimdio.pdf
\caption/f Obrázek ukazuje čtení z registru po komunikaci po seriové lince STA \cite[IEEE8023].
\medskip

Za pomoci STA, MAC může vyčítat z registrového prostoru PHY. Část registrového prostoru je definována tež standardem 802.3.
Registrový prostor obsahuje dva základní sety registrů kontrolní a stavový.
Všechny PHY, které podporují MII by měli používat tuto sadu registrů \ref[miireg].


\medskip
\clabel[miireg]{Registry STA rozhraní.}
%\picw=13cm \cinspic MIItoISO.png
\picw=15cm \cinspic images/802_miimdio.pdf
\caption/f Obrázek popisující registry STA rozhraní.   Obrázek je převzat z~\cite[IEEE8023].
\medskip


Stavový a kontrolní registrové prostory můžeme vidět v příloze \ref[MIIcntr] a \ref[MIIstat].
Tyto registry specifikují základní vlastnosti pro 100Mb/s a 1Gb/s PHYs.
Registry 2-14 jsou částí rozšířeného regisrového prostoru \cite[IEEE8023] (22.2.4).


% Ne do DP
%\sec GMII

%The Gigabit Media Independent Interface (GMII) is similar to the MII. The GMII uses the MII management interface and register set specified in 22.2.4. These common elements of operation allow Station Management to determine PHY capabilities for any supported speed of operation and configure the station based on those capabilities. In a station supporting both MII and GMII operation, configuration of the station would include enabling either the MII or GMII operation as appropriate for the data rate of the selected PHY. Most of the MII and GMII signals use the same names, but the width of the RXD and TXD data bundles and the semantics of the associated control signals differ between MII and GMII operation. The GMII transmit path clocking also differs significantly from MII clocking. MII operation of these signals and clocks is specified within Clause 22 and GMII operation is specified within Clause 35.
%GMII obsahuje 3ti zakladni registr Extended status reg(15)

\secc Reduce Media Independet Interface


Reduce Media Independet Interface (RMII) je obdoba MII, která má snížený počet pinů z 16 na 8 \cite[RMII].
RMII je nízkonákladova obdoba MII, která přidává pouze reconciliation layer (RS) na oboch stranách MII či náhradu aktualního RS v MII \cite[RMIIcon].

Rozšířenou RS můžeme vidět na obrázku \ref[RMIIRS]


\medskip
\clabel[RMIIRS]{Rozšíření MII o další RS.}
%\picw=13cm \cinspic MIItoISO.png
\picw=15cm \cinspic images/rmiirs.pdf
\caption/f Obrázek, který ukazuje jak rozšíření MII o další vrstvu RS vypadá.  Obrázek je převzat z~\cite[RMII].
\medskip

RMII používá stejné menežovací rozhraní jako MII definované IEEE 802.3u \cite[IEEE8023] popsané \ref[MDIOs].
Datové vodiče RMII jsou pouze 2 bity široké.
Dále je jsou sjednoceny signály COL a CRS do jednoho signálu CRS\_DV.
Detekce kolizí na MAC je logický součin signálů TX\_EN a CRS\_DV.
Zdrojem hodinového signálu je vrstva MAC nebo je generován externími hodinami.

%TODO pridat obrazek RSl

% TODO
%\sec linux

%\sec info kernel

%\sec loading proces

%\sec disribuce

%\secc openwrt

\medskip \medskip \medskip \medskip

\sec Subsystém operačního systému Linux

Tato kapitola obsahuje informace o tom, co to je ovladač. A jak je ovladač přepínače \ref[DSA] a usb zařízeních integrovaný \ref[USBst]  v operačním systému Linux.

\secc Co je to ovladač

Ovladač je kus programu (algoritmu), který zpravuje nebo kontroluje určité zařízení připojené k počítači.
Ovladače vytvářejí softwarové rozhraní pro hardwarové zařízení, které umožňují operačnímu systému ovládat tento hardware aniž by uživatel věděl jak přesně funguje \ref[LDD3].

\secc device

%{\em struct} 
%\fnote{\url{Odkaz na IEEE}}}.

Každé zařízní na té nejnižší úrovni je v operačním systému Linux je reprezentováno instancí struktory {\em struct device}.
Struktora device obsahuje informace o tom co zařízení potřebuje aby správně fungoval v systému.
Mnoho subsystému uchovává informace o této struktuře, na které stavějí.
Je velmi optížné dnes narazit na zařízení, které je reprezentováno pouze struktorou {\em struct device}.
Místo toho  některé struktury v {\em struct device}, jako je kobj, obecně implementují vyšše reprezentovaná zařízení \ref[DEV].
\fnote{Popis datové struktury {\em struct device} můžete najít na \url{http://lxr.free-electrons.com/source/include/linux/device.h\#L730}}.


\secc mii bus

{\em Struct mii\_bus je struktura, která reprezentuje MII rozhraní a umožňuje přístup k registrům popsaným \ref[miireg].

\label[DSA]

\secc Distributed Switch Architecture

Distributed Switch Architecture (DSA) je rozhraní pro menežování hardwarových přepínačů \cite[DSA].
%Obsahuje MII menežovací registry, příkazy pro nastavení přepínače a formát ethernetových hlaviček, který signalizuje na kterém portu byl paket přijat nebo ktery se hodlá odeslat 

Tento driver podporuje přepínače, které jsou připojeny způsobem zobrazeným na obrázku \ref[DSAcon].

\medskip
\clabel[DSAcon]{Připojení DSA k CPU.}
%\picw=13cm \cinspic MIItoISO.png
\picw=15cm \cinspic images/DSAcon.pdf
\caption/f Obrázek ukazuje jak je možné připojit DSA přepínač k procesoru. Obrázek je převzat z~\cite[DSA].
\medskip


Tento ovladač přepínače reprezentuje každý port jako oddělené síťove rozhraní \ref[NETDEV].
Pollování, stav portů, MII menežment rozhraní je proveden za základě rozhraní ethtool \fnote{Ethtool je standartní utilita Linuxu pro ovládání a podporu ovladačů a zařízení.}.


DSA podporuje i propojení mezi switchy tak jak můžete vidět na obrázku \ref[DSAcon2], které pak mužem také ovládat.

\medskip
\clabel[DSAcon2]{Připojení více přepínačů DSA k CPU.}
%\picw=13cm \cinspic MIItoISO.png
\picw=15cm \cinspic images/DSAcon2.pdf
\caption/f Obrázek ukazuje jak je možné připojit DSA přepínač k procesoru. Obrázek je převzat z~\cite[DSA].
\medskip


\label[NETDEV]

\secc net device

Struktura {\em struct net\_device } je struktura popisující síťové rozhraní.
Síťové rozhraní může příjímat a vysílat velké množství dat a proto je implementace podobná diskovým zařízením \ref[LDD3].
Síťová zařízení přijímají data asynchroně ukládájí je do vyrovnávacích pamětí.

V linuxovém jádře je síťový subsystém naimplementován tak, že je protokolově nezávyslý.

\label[USBst]

\secc usb driver

Linuxové jádro podporuje dva druhy USB zařízení ovladače na hostovském systému a ovladače na zařízení \cite[LDD3].
USB ovladač na hostovském systému, jak může být podle názvu jasné, běží systému ke kterému je USB zařizení připojeno.
Ovladače na zařízení neboli \"USB gadget drivers\"  zařízení připojojené k hostovskému systému vypadá jako hostovská stanice připojena přez USB.
Na obrázku níže mužeme vidět USB stack, kde USB muže existovat v několika různých subsystémech (net, block, char ...). USB core dává rozhraní pro USB ovladače, které chtějí kontrolovat a přistupovat k hardwaru.


Koncové body USB jsou popsány v jádře Linuxu strukturou {\em struct usb\_host\_endpoint}.
Tato struktora obsahuje informace o realném koncovém bodu v struktuře {\em struct usb\_endpoint\_descriptor}.
Struktora {\em struct usb\_endpoint\_descriptor} popisuje data pomocí příjmutých dekriptorů.

USB rozhraní je složeno z několik koncových bodů, které tvoří jedno logické připojení jako je napřiklad myš, klávesnice, video atd. Některá USB zařízení mohou mít více rozhraní.
USB repoduktory mohou mít dvě rozhraní jako je USB klávesnice pro tlačítka a USB audio stream \ref[LDD3].%TODO presunout nahoru
Pro každé rozhraní Linuxové jádro používá jeden hardwarový ovladač.
Toto rozhraní je popsáno strukturou {\em struct usb\_interface}.
Tato struktura je to co USB core předává USB ovladačům.
Každé zařízení je vázáno konfigurací a jedno USB zařízení muže mít těchto konfigurací více.
Tyto konfigurace se mouhou v průběhu běhu zařízení měnit.
A však Linuxové jádro nemůže obsluhovat více konfigurací v jeden čas. Linuxové jádro popisuje konfigurace v struktuře {\em struct usb\_host\_config} a celé USB zařízení strukturou {\em struct usb\_device}.
Ovladače USB zařízení  obyčejně přepisují data z {\em usb\_interaface} do {\em struct usb\_device}.

Samotné struktury {\em struct usb\_device} a {\em struct usb\_interface} jsou zobrazeny v sysfs\ref[SYSFS] jako jednotlivá soubory.

Na řádce níže můžeme vidět ukázkovou cestu k {\em struct usb\_device} reprezentovanou pomocí sysfs:

\begtt  /sys/devices/pci0000:00/0000:00:09.0/usb2/2-1  \endtt

K výše uvedené cestě k USB zařízení je na řádku níže zvírazněno USB rozhraní reprezentováno strukturou {\em struct usb\_interface} je pojmenovan podle formátu root\_hub-hub\_port:config.interface (pro hlubší stromy je schéma root\_hub-hub\_port-hub\_port:config.interface):

\begtt /sys/devices/pci0000:00/0000:00:09.0/usb2/2-1/2-1:1.0 \endtt

Pro komunikaci se všemi USB zařízeními Linuxové jádro používá USB request block (urb). Urb se používá k posílaní a příjmu dat pro specifický koncový bod USB a specifické USB zařízení asynchroním způsobem.

%\secc net\_device XXX

\label[USBNET]

\secc usbnet

Usbnet je subsystém Linuxového jadra, který umožňuje ovládat USB-síťové zařízení jako je ethernet, DSL, IDSN atd \ref[USBNET].
Usbnet je to genericky USB síťový framework, který pracuje na různých rychlostech a nad různými protokoly.


\secc System Filesystem

SYstem FileSystem (Sysfs) je charakteristika Linuxového jádra od verze 2.6, která umožňuje kernelovému kódu exportovat informace do userspace za použití paměti VFS\ref[SYSFS].
Hlavním učelem je reprezentovat objekty, jejich atributy a jejich vstahy navzájem.
Většina atributů je reprezentována souborem jež je ve formátu ASCII a obsahuje pouze jednu hodnotu.
Sysfs poskytuje dvě složky:

\begitems
*kernel programing interface -- Slouží k exportování a importování položek (viz níže) skrz sysfs do kernelu
*user interface -- Slouží k vidění a manipulaci těchto položek (viz níže), které mapuje zpět na objekty v kernelu
\enditems

Mapování mezi objekty můžeme vidět v \ref[SYSFSmap].

\midinsert \clabel[SYSFSmap]{Mapováni objektů a atributu v sysfs.}
\ctable{ll}{
 \hfil  Interni   & Externi  \crl \tskip4pt
        Kernelovské objekty    & Složky  \cr
        Atributy objektů     & Soubory  \cr
        Vazby mezi objekty   & Symbolické odkazy  \cr
}
\caption/t  Tabulka prevzana z \cite[SYSFS].
\endinsert

V systemu můžeme vidět tyto zařízení vidět ve složce  {\em /sys/ }.
Složka /sys/ může vypadat následovně:
\begtt
    /sys/
    |-- block
    |-- bus
    |-- class
    |-- devices
    |-- firmware
    |-- module
    ‘-- power
\endtt

Sysfs je reprezentováno v Linuxovém jádře jako struktura {\em struct kobj}, který obsahuje tři druhu obslužného volání na zápis, čtení a na uvolňení.

\chap Použitý hardware

\label[T20]

\sec Colibri T20

Colibri T20 je počítačový modul postaven na základě NVIDIA Tegra 2 embeded systému na čipu (SOC) \cite[SOC].
SOC modul Tegra 2 je založen na dvoujádrovém procesoru Cortex A9 se symetrickým procesorovým jádrem od firmy ARM s rychlostí okolo 1 Ghz.
Jelikož se jedná o SOC obsahuje tento čip také mnoho bluků jako je například Audio/Video rekordér.
Grafický čip s podporou 2D rendrovaní a 3D pixel a vektor shadrem.

Čip obsahuje periferie popsané na blokovém diagramu na obrázku \ref[SOC].

\medskip
\clabel[SOC]{Colibri T20 blokový diagram.}
\picw=15cm \cinspic images/soc.pdf
\caption/f Obrázek popisuje části integrované na modulu Colibri T20 \cite[SOC].
\medskip

Toradex dodává podporu BSP pro Linuxové jádro 3.1.

%chtelo by to swap Colibri a ARM ci spojit...
\sec NVIDIA Tegra 2

Dvoujadrový procesor ARM Cortex A9 symmetric MPCore je jeden z aplikačních procesorů vyvíjený firmou ARM.
Tento ARM je 32-bitová RISC architektura vyvíjená společností ARM Holding věnující se vývoji procesorových jader.
Tento procesor porporuje out-of-order a spekulativní provádění instrukcí.
Má plnou podporu koherence pamětí pro symetrické procesory.
Toto jádro dosahuje vysokého grafického výkonu, ale také výpočetního výkonu viz. \cite[BENCH].
Procesor též obsahuje jednotku pro výpočty s plovoucí řádovou čárkou.

Popis jádra NVIDIA Tegra 2 můžeme vidět na obrázky níže \ref[TEGRA].

\medskip
\clabel[TEGRA]{Popis jádra NVIDIA Tegra 2.}
\picw=15cm \cinspic images/tegra2.pdf
\caption/f Obrázek popisuje vnitřek proceosru NVIDIA Tegra 2 \cite[TEGRA].
\medskip

\label[ASIX]

\sec Asix AX88772b


AX88772b je zákaznický integrovaný obvod, který umožňuje plug-and-play Fast Ethernet internetové připojení pro zařizení používající stantardizované USB \ref[USB].

Jako jediný z výrobních řad dosahuje revize AX88772bli teplotní rozsah od -40 stupňů Celsia až +85 stupňu Celsia.

AX88772b má modifikovatelnou vícefunkční sběrnici, která umožňuje připojit RMII \ref[RMII] nebo jako Reverse-RMII pro MAC-to-MAC připojení k nějakému mikrokontroléru s ethernet MAC RMII rozhraním.

MAC rozhraní je plně kompatibilní se standardy IEEE~802.3, IEEE~802.3u.
%neudelat 802.3 chap?

%TODO ASIX
%\medskip
%\clabel[ASIXcip]{.}
%\picw=15cm \cinspic images/asix.pdf
%\caption/f Obrázek popisuje  \cite[ASIX].
%\medskip
%jake umoznuje propojeni po rmii?

Jak je vidět na obrázku \ref[ASIXcip] čip AX88772b obsahuje ethernetovou PHY, která nemůže být aktivní současně s RMII rozhraním.

\label[MARVELL]

\sec Marvell 88E6065

Marvell 88E6065 je 5+1 portový ethernetový přepínač integrovaný na jednom čipu.
Obsahuje  pět portů na fyzicke vrstvě 10~BASE-T/100~BASE-TX, znichž dva porty můžou být použity pro optickou linku 100~BASE-FX.
Šestým portem je nezávyslí Fast ethernet MAC.
Zařízení má high-speed, non-blocking čtyřúrovňový QoS.
PHY podporuje plag-n-play s možností Auto-Crossoer, Auto-polarity a Auto-nogotiation.
Zařízení dále odporuje 64 z 4096 802.1Q WLAN s 3 úrovňovou ochranou.
Má dva RMII/MII/SNI rozhraní, která mohou být připojena k menežovacímu rozhraní nebo ke kontroleru s integrovanou MAC.

MAC a PHY rozhraní jsou plně kompatibilní se standardy IEEE~802.3, IEEE~802.3u a IEEE~802.3x.
Zařízení může být konfigurováno přez STA nebo může nastavení načítat z EEPROM paměti.

Výjmečně se může Marvell 88E6065 využívat podpory pro směrovače a výchozí brány.
%jake umoznuje porjojeni RMII

\label[SWITCH]

\sec Menežovatelný přepínač


Menežovatelný přepínače umožňuje konfiguraci ethernetoveho přepínače za pomoci USB kontroleru Asix AX88772b od firmy Asix \ref[ASIX], ke kterému je připojena externí pamět eeprom M93-C66.
Tento kontrolér je připojen RMII rozhraním k čipu Marvell 88E6065 \ref[MARVELL].
Marvell 88E6065 je ethernetový přepínač, který má k sobě kromě zmíněného RMII portu připojeno pět ethernetových portů a taktéž je kněmu připojena externí pamět eeprom.
Oba tyto čipy mají též vlastní krystal na frekvenci 25~Mhz.
Deska je taktéž osazena regulátorem napětí.

Desku menežovatelného přepínače je vyrobena firmou Retia a.s. \fnote{Firma zabívající se elektronickými vojeskými systémy, zaznomovými zařízeními a lokalizačních a bezpočnostních systémů. \url{www.retia.cz}} na základě jejich návrhu a zapojení popsaných v přílohách \ref[zapojeni] \ref[zapojeni2] \ref[zapojeni3].

\chap Implementace

Tato kapitola rozebírá update systému pro modul Colibri T20 \ref[T20], který byl zapotřebí pro další implementaci ovladačů pro aktuální Linuxové jádro.
Dále se zabívá způsobem oživení desky \ref[SWITCH] a následným popisem tvorby ovladače pro menežovatelný přepínač.

Vývoj probíhal na vyvojovém kitu s modulem Colibri T20.
Pro vývoj ovladače bylo zvoleno jádro z upstreamové větve kernelu
\fnote{\url{http://git.kernel.org/cgit/linux/kernel/git/torvalds/linux.git}}
konkrétně se začlo s verzí 3.17 a došlo až k přeportování ovladače na verzi 3.19.
Portace upstreamového jádra je popsána v \ref[UPSTREAM].

\label[UPSTREAM]

\sec Portace upstreamové linuxové jádro a zavaděče

Portace upstramového jádra vyžaduje několik nástrojů a programů.

\secc U-Boot

V první řadě je potřeba naistalovat zavaděč, který se postará o zavedení Linuxové jádra.
Pro tento úkol byl zvolen U-Boot, který používá již výrobce desky Colibri T20.
Jen byla zvolena jeho novejší verze
\fnote{U-Boot je možný stáhnout z repozitáře \url{git://git.denx.de/u-boot.git}}.

Potup nahrání nového U-Bootu je popsán v příloze \ref[UPUBOOT].

\secc Linuxové jádro

V dalším krokem byla instalace současné verze jádra, která byla nutná k vývoji ovladačů a komunikací s linuxovou komunitou.
Jelokož upstreamové jádro nepodporuje NAND paměť načítání nového jádro probýhalo přez síť \ref[UPLINUX].


\sec Rozchozeni hardwaru

Rozchozeni hardwaru byla jedna z nejobtiznejsich casti.

Pricemz se zacalo s testovanim USB kontroleru Asix AX88772b. A po rozchozeni komunikacnich interfacu bylo pokracovano s ladenim ethernetoveho prepinace 88E6065.
Posledni casti bylo testovani ethernetovych portu.
Pricemz bylo objeveno nekolik chyb v navrhu zarizeni. Ktera na zaklade mojich pripominek byla opravena(XXX to zaveru).

\secc Oživení Asix AX88772b

Při oživování USB kontroléru byla objevena zavada na jeho připojení na desce.
Tato závada se projevovala chybnou komunikací na USB sběrnici, která byla objevena při použití modulu Colibri T20.
Tu to chybnou komunikaci můžete vidět na obrázku \ref[usbbuserr].
Kde je znatelně vidět, že pakety přijímalo zeřízení poškozené.

\medskip
\clabel[usbbuserr]{Chybná komunikace na USB sběrnici.}
%\picw=13cm \cinspic MIItoISO.png
\picw=15cm \cinspic images/wrong_com_wire.pdf
\caption/f Na obrázku můžeme vidět chybnou komunikaci na usb sběrnici mezi PC a USB kontrolerem Asix AX88772b zaviněnou nepřipojením blokovacího kondenzatoru.
\medskip

Po úpravě zapojení kontroléru. Začal USB kontroler odpovídat na dotazy.

Bylo zvoleno zapojení, které aktivuje RMII rozhraní, nastavením spojky K2 na zapojení \ref[zapojeni3].

%TODO 
%\medskip
%\clabel[usbbuserr]{Chybna komunikace na usb sbernici.}
%\picw=15cm \cinspic images/wrong_com_wire.png
%\caption/f Na obrázku .
%\medskip

Dále bylo potřeba nastavit vhodne chovani čipu Asix, aby začal fungovat rozhraní RMII, které je blíže popsané v \ref[ZPRAC].

\secc Oživení ethernetového přepínače Marvell 88E6065

K oživení a stabilnímu chování ethernetového přepínače Marvell 88E6065 bylo zapotřebí aktivovat RMII rozhraní a provést několik zásahů do zapojení čipu na desce.
Byli doplněny chybějící blokovací kondenzátory.
Z důvodu nestabilních vstupních hodnot na čipu.
Byl doplněn kondenzátor, který zařízení vyresotovával čip po dobu časové konstanty RC po připojení přepínače k napájení.
A na základě mojich požadavků byli přidány pull-up a pull-down rezistory pro správné a deterministické chováni přepínače.

\sec Návrh ovladače

Návrh ovladače byl založen na propojení ovladače pro Asix AX88772b a subsystému DSA \ref[DSA].
TODO asi obrazky struktur + nejaky popis

\label[ZPRAC]

\sec Zpracování

Sekce zpracování popisuje jednotlivé časti ovladače.

Jelikož nebyl ovladač pro zařízení Asix AX88772b připraven na připojení RMII rozhraní.
Bylo potřeba nastavit tento interface jako aktivní a však tento interface nemůže být aktivní současně s ethernet PHY, která je taktéž integrovaná na čipu \ref[ASIX].
Implementace oladače je popsána v \ref[asixdri].
Nastalo několik dotazů jak vytvořit připojení přepínače. Zdali ho detekovat automaticky či přez userspace API, konktrétně sysfs.

Dále bylo potřeba upravit subsystém DSA, aby podporoval hotswap zařízení popsaný v \ref[dsadri].

Poslední částí bylo napsat samotný ovladač pro Marvell 88E6065, aby podporoval současné zařízení popsané \ref[mvdri].

\label[asixdri]

\secc Ovladač zařízení Asix AX88772b

Asix AX88772b má již podporu v Linuxovém jádře a však pouze na jedno z jeho rozhraní.
Tím rozhraním je ethernet PHY \ref[ASIX].
Bylo potřeba doplnit ovladač o aktivaci RMII rozhraní.
A byla zvolena možnost, že aktivace RMII rozhraní proběhne přez zápis do souboru {\em dsa\_bind}, jež je součástí subsystému sysfs, a je umístěn ve složce USB zařízení.
Po zapsání do tohoto souboru se vyvolá obslužná rutina objektu {\em struct kobj usb\_dsa\_store() }, která zavolá {\em dsa\_bind()}
, která nastaví příslušné registry pro aktivování RMII rozhraní.
Nadále tato rutina též otestuje zdali, je k USB kontroléru připojeno zařízení Marvell 88E6065, dle standardizovánému registrovému prostoru.
Konkrétně vyčtením z třetího registru z \ref[miireg], která obsahuje ID a revizy zařízení.
Tato část obslužné rutiny též vyplní strukturu {\em struct asix\_common\_private}, které ve verzi kompatibilní s jadrem verze 3.17 vypadala následovňě:

\begtt
struct asix_common_private {
   struct asix_rx_fixup_info rx_fixup_info;
#ifdef CONFIG_NET_DSA
   struct kobject kobj;
   struct mii_bus *mdio;
   int use_embphy;
   bool dsa_up;
   struct usbnet *dev;
#endif
}
\endtt

Ve verzi linuxového jádra 3.19 subsystém DSA nevyžaduje {\em struct mii\_bus} a byl odstraněn a nahrazen {\em struct platform\_device}, aby se vyhnulo většině změn v subsystému DSA dle připomýnky v mailing listu \ref[ML].
A proto ovladač zařízení Asix AX88772b byl doplněn o funkci {\em dsa\_create\_pdev}, která zajistí naplnění {\em struct platform\_device}.
Tato struktura se vytváří při po kontrole zařízení ve funkci {\em dsa\_bind}.

Po naplnění základních struktur ovladač Asix AX88772b zavolá {\em dsa\_probe\_netdevice()}, který vytvoří softwarové rozhrání přepínače. Změna DSA subsystému je popsaná v \ref[dsadri].

\label[dsadri]

\secc Úprava subsystému DSA

Subsystém DSA byl rozšířen o veřejnou funkci {\em dsa\_probe\_netdevice()}, která vytváří softwarové rozhraní pro přepínač.
Každý port přepínače je reprezentován vlastním ethernetovým rozhraním net\_device \ref[NETDEV], které mají společného rodiče stejného typu.

Bylo potřeba odloučit DSA od platform\_driveru což následně zapříčiňuje rozšíření dsa\_driver \ref[mvdri].

Větší rozsah změn čekala uvolňovací část ovladače.
Jelikož se nepředpokladá, že by se tento ovladač někdy uvolňoval.
Tato funce nebyla dodělána a proto byla upravena {\em dsa\_destroy()}, která v současné verzi podporuje uvolnění všech portů a interních struktur dsa ovladače jako je {\em struct dsa\_switch a struct dsa\_switch\_tree}.


\label[mvdri]

\secc Ovladač zařízení Marvell 88E6065

% popis ovladacove struktury
% jednotlivych funci
% TODO


\sec celek?

\sec Odezva komunity

Odezva komunity byla velmi rychlá po odeslaní RFC patche
\fnote{\url{http://lists.openwall.net/netdev/2015/04/21/20}}.
se mi dostalo odpovědi během pouhých pár hodin po jeho odesalní.
Kde Andrew Lunn se vyjádřil k mému nápadu a navrhl nějaké optimalizace k mému řešení.
Také se strhla debata o možnostech DSA, že DSA není připravené pro podporu hotswap zařízení.
A pravděpodobně tato změna nenastane ani v příštích verzích Linuxového jádra.

\chap Otestujte implementované řešení zejména s ohledem na propustnost sítě a stabilitu

Tato kapitola se zabiva

\begtt
[11.046] Unable to handle kernel paging request at Virtual Addr 6b6b6b77
[11.053] pgd = c0004000
[11.055] [6b6b6b77] *pgd=00000000
[11.059] Internal error: Oops: 5 [#1] PREEMPT SMP ARM
[11.064] Modules linked in:
[11.067] CPU: 0 PID: 0 Comm: swapper/0 Not tainted 3.19.0 #298
[11.075] Hardware name: NVIDIA Tegra SoC (Flattened Device Tree)
[11.082] task: c08ba2a8 ti: c08ae000 task.ti: c08ae000
[11.087] PC is at get_next_timer_interrupt+0xa0/0x294
[11.092] LR is at get_next_timer_interrupt+0x60/0x294
[11.098] pc : [<c006f474>]    lr : [<c006f434>]    psr: 90000093
[11.098] sp : c08aff00  ip : 6b6b6b6b  fp : 00000000
[11.109] r10: c0927540  r9 : ffff8f21  r8 : ffff8f20
[11.114] r7 : 00000000  r6 : 3fff8f1f  r5 : ffff8f20  r4 : ffff8f17
[11.121] r3 : 6b6b6b6b  r2 : 00000021  r1 : 00000023  r0 : c0927674
[11.127] Flags: NzcV IRQs off FIQs on Mode SVC_32 ISA ARM Segment kernel
[11.135] Control: 10c5387d  Table: 1ac0404a  DAC: 00000015
[11.140] Process swapper/0 (pid: 0, stack limit = 0xc08ae238)
[11.146] Stack: (0xc08aff00 to 0xc08b0000)
[11.151] ff00: c08affa4 c060ecc0 0000001d c00625f0
    fe44010c 0000000d c08aff40 c08b7154
[11.159] ff20: 755c2278 00000002 c08ac640 60000000
    00000185 dbbb4280 00000000 ffff8f20
[11.216] [<c006f474>] (get_next_timer_interrupt)
    from [<c007e34c>] (__tick_nohz_idle_enter+0x2d4/0x41c)
[11.226] [<c007e34c>] (__tick_nohz_idle_enter)
    from [<c007e4e0>] (tick_nohz_idle_enter+0x30/0x6c)
[11.235] [<c007e4e0>] (tick_nohz_idle_enter)
    from [<c00552c4>] (cpu_startup_entry+0x18/0x278)
[11.244] [<c00552c4>] (cpu_startup_entry)
    from [<c0868c0c>] (start_kernel+0x360/0x3cc)
\endtt

\sec propustnost

\sec porovnani s jinym eth interfacem

\sec konfigurace 

co se testuje a jak se testuje(co pisu za prikazy) -> vysledky

\sec logterm

\chap Zaver

funguje, odezva, linuxova komunita, dosazene vysledky, vykon
MII,RMII,GMII,RGMII,SGMII,QGMII,XAUI
