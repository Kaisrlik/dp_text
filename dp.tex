\def\ctustyle{{\tenss CTUstyle}}
\def\ttb{\tt\char`\\} % pro tisk kontrolních sekvencí v tabulkách

\input ctustyle
\worktype [M/CZ]
\faculty {F3}
\department {Katedra Řídící techniky}
\title {Integrace Ethernetového přepínače do~embedded zařízení s~OS Linux}
\author {Jan Kaisrlík}
\supervisor {Ing. Michal Sojka Ph.D.}
\date {Květen 2015}
\abstractEN {XXX}
\abstractCZ {
Cílem této diplomová práce je portace systému Linux na modul Colibri T20 a implementace ovladače do operačního systému Linux pro zařízení na bázi přepínacího obvodu 88E6065 vyvíjené firmou Marvell, který je připojen USB kontrolérem AX88772b od firmy Asix, které splňuje industriální rozsah teplot.
Tato práce se též zabývá MII rozhraním, které slouží k připojení přepínacích obvadů, směrovacích zařízení a fyzických vrstev.
A také se zabývá USB sběrnicí.
Práce též rozebírá subsystémem ethernetového zasobníku a přepínacích obvodů v operačním systému Linux.
}
\declaration {
Prohlašuji, že jsem svou diplomovou práci vypracoval samostatně a~použil jsem pouze poklady uvedené v~přiloženém seznamu.

Nemám závážný důvod proti použití tohoto školního díla ve~smyslu §~60 Zákona č.121/2000~Sb., o~právu autorském, o~právech souvisejících s~právem autorským a~o~změně některých zákonů (autorský zákon).

   V~Praze~dne~11.5.2015
   \signature 
}
\keywordsCZ{Linux, Drivers, USB, DSA, Colibri T20, Marvell mv88e6065, Asix ax88772b}
\keywordsEN{Linux, Ovladače, USB, DSA, Colibri T20, Marvell mv88e6065, Asix ax88772b}

\thanks {}
\draft
\linespacing=1.5
\makefront
\chyph


\input uvod

\input prilohy

\bye
\endtt
