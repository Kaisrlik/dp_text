\def\ctustyle{{\tenss CTUstyle}}
\def\ttb{\tt\char`\\} % pro tisk kontrolních sekvencí v tabulkách

\input ctustyle
\worktype [M/CZ]
\faculty {F3}
\department {Katedra Řídící techniky}
\title {Integrace Ethernetového přepínače do~embedded zařízení s~OS Linux}
\author {Jan Kaisrlík}
\supervisor {Ing. Michal Sojka Ph.D.}
\date {Květen 2015}
\abstractEN {XXX}
\abstractCZ {

Cílem této diplomové práce je portace operačního systému Linux na modul Colibri T20 a implementace ovladače do tohoto operačního systému pro teplotně odolný Ethernetový přepínač.
Tento přepínač je postaven na obvodu 88E6065 vyvinutým firmou Marvell, ke kterému je připojen USB-Ethernet kontrolérem AX88772b od firmy Asix, který splňuje požadovaný průmyslový rozsah teplot.
Tento kontrolér slouží k připojení přepínače za pomoci USB sběrnice.

Dané zadání bylo vyřešeno rozšířením stávajícího ovladače pro kontrolér AX88772b, úpravou přepínacího subsystému pro podporu zařízení připojených za běhu systému a vytvořením ovladače pro přepínací obvod 88E6065.

V práci byl vytvořen systém, který úspěšně zprovozní přepínač a vytvoří pro něj rozhraní viditelné v uživatelském prostoru operačního systému.

}
\declaration {
Prohlašuji, že jsem předloženou práci vypracoval samostatně a že jsem uvedl veškeré použité informační zdroje v souladu s Metodickým pokynem o dodržování etických principů při přípravě vysokoškolských závěrečných prací.
\medskip
   V~Praze~dne~11.5.2015
   \signature 
}
\keywordsCZ{Linux, Ovladače, USB, DSA, Colibri T20, Marvell mv88e6065, Asix ax88772b}
\keywordsEN{Linux, Drivers, USB, DSA, Colibri T20, Marvell mv88e6065, Asix ax88772b}

\thanks {}
\draft
\linespacing=1.5
\makefront
\chyph


\input uvod

\input prilohy

\bye
\endtt
