\label[UPUBOOT]

\sec Nahrátí upstreamového zavaděče U-Boot

git clone git://git.denx.de/u-boot.git

Utility tegrarcn a cbootimage sloužící k flashování:

 git clone https://github.com/NVIDIA/cbootimage-configs.git

BCT + U-Boot:

 cbootimage -gbct -t20 tegra20/toradex/colibri-t20/colibri-t20\_512\_v12\_nand.bct.cfg colibri-t20\_512\_v12\_nand.bct
 cp ../u-boot/u-boot-dtb-tegra.bin u-boot.bin
 cbootimage -t20 tegra20/toradex/colibri-t20/colibri-t20_512_v12_nand.img.cfg colibri-t20\_512\_v12\_nand.img

Nahrání image v recovery módu a následné spustění image:

 tegrarcm --bct colibri-t20\_512\_v12\_nand.bct --loadaddr=0x00108000 --bootloader=../u-boot/u-boot-dtb-tegra.bin

Přehrání bootloaderu v U-Bootu:

Tegra20 (Colibri) # usb start
(Re)start USB...
USB1: USB EHCI 1.00
scanning bus 1 for devices... 1 USB Device(s) found
USB2: USB EHCI 1.00
scanning bus 2 for devices... 2 USB Device(s) found
 scanning usb for storage devices... 0 Storage Device(s) found
 scanning usb for ethernet devices... 1 Ethernet Device(s) found
Tegra20 (Colibri) # setenv ipaddr 192.168.80.90; setenv serverip 192.168.80.3 
Tegra20 (Colibri) # tftpboot 0x02100000 colibri-t20\_512\_v12\_nand.img
Waiting for Ethernet connection... done.
TFTP from server 192.168.80.3; our IP address is 192.168.80.90
Filename \'colibri-t20\_512\_v12\_nand.img\'.
Load address: 0x02100000
Loading: ##################################################
         3.6 MiB/s
done
Bytes transferred = 729088 (b2000 hex)
Tegra20 (Colibri) #

A následné přepsání v NAND paměti:

Tegra20 (Colibri) # nand erase.chip

NAND erase.chip: device 0 whole chip
Erasing at 0x3ffc0000 -- 100\% complete.
OK
Tegra20 (Colibri) # nand write 0x02100000 0 0xb2000

NAND write: device 0 offset 0x0, size 0x95000
 577536 bytes written: OK

0xb2000 délka popsána u přenosu
