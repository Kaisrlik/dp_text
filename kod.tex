\label[UPUBOOT]

\sec Nahrátí upstreamového zavaděče U-Boot

Stáhnutí a přeložení zdrojových kódů převaděče \ref[UPLINUXT20].

\begtt
$ git clone git://git.denx.de/u-boot.git
$ export CROSS_COMPILE=~/gcc-linaro/bin/arm-linux-gnueabihf-
$ export ARCH=arm
$ make colibri_t20_iris_config
$ make -j8
\endtt

Utility tegrarcn a cbootimage sloužící k flashování:

\begtt
$ git clone https://github.com/NVIDIA/cbootimage-configs.git
\endtt

Překlad BCT a vytvoření image s U-Boot binárkou:

\begtt
$ cbootimage -gbct -t20 
		itegra20/toradex/colibri-t20/colibri-t20_512_v12_nand.bct.cfg 
 		colibri-t20_512_v12_nand.bct
$ cp ../u-boot/u-boot-dtb-tegra.bin u-boot.bin
$ cbootimage -t20
 		tegra20/toradex/colibri-t20/colibri-t20_512_v12_nand.img.cfg 
 		colibri-t20_512_v12_nand.img
\endtt

Nahrání image v recovery módu a následné spustění image:

\begtt
$ tegrarcm --bct colibri-t20_512_v12_nand.bct --loadaddr=0x00108000
 		--bootloader=../u-boot/u-boot-dtb-tegra.bin
\endtt

Nahrátí bootloaderu do paměti modulu v U-Bootu konzoli:

\begtt
Tegra20 (Colibri) # usb start
(Re)start USB...
USB1: USB EHCI 1.00
scanning bus 1 for devices... 1 USB Device(s) found
USB2: USB EHCI 1.00
scanning bus 2 for devices... 2 USB Device(s) found
 scanning usb for storage devices... 0 Storage Device(s) found
 scanning usb for ethernet devices... 1 Ethernet Device(s) found
Tegra20 (Colibri) # setenv ipaddr 192.168.80.90
Tegra20 (Colibri) # setenv serverip 192.168.80.3
Tegra20 (Colibri) # tftpboot 0x02100000 colibri-t20_512_v12_nand.img
Waiting for Ethernet connection... done.
TFTP from server 192.168.80.3; our IP address is 192.168.80.90
Filename 'colibri-t20_512_v12_nand.img'.
Load address: 0x02100000
Loading: #################################
         3.6 MiB/s
done
Bytes transferred = 729088 (b2000 hex)
\endtt

Délka přeneseného obrazu je 0xb2000 viz. výše. ta bude potřeba k zapsání na NAND.
Přepsání bootloaderu v NAND paměti:

\begtt
Tegra20 (Colibri) # nand erase.chip

NAND erase.chip: device 0 whole chip
Erasing at 0x3ffc0000 -- 100% complete.
OK
Tegra20 (Colibri) # nand write 0x02100000 0 0xb2000

NAND write: device 0 offset 0x0, size 0x95000
 577536 bytes written: OK
\endtt

\label[UPLINUX]

\sec Nahrátí upstreamového linuxového jádra

Stážení a přeložení zdrojových kódů Linuxového jádra.

\begtt
$ git clone git.kernel.org/pub/scm/linux/kernel/git/torvalds/linux.git
$ export LOADADDR=0x408000
$ export ARCH=arm
$ export CROSS_COMPILE=/PATH/gcc-linaro/bin/arm-linux-gnueabihf-
$ cd linux
$ git checkout v3.19
$ make tegra_defconfig
$ make -j8 uImage
\endtt

Náhrátí linuxového jádra na desku za pomoci ethernetového připojení.

\begtt
Tegra20 (Colibri) # usb start
Tegra20 (Colibri) # setenv ipaddr 192.168.80.90
Tegra20 (Colibri) # setenv serverip 192.168.80.3
Tegra20 (Colibri) # tftpboot ${kernel_addr_r} zImage
Tegra20 (Colibri) # tftpboot ${fdt_addr_r} tegra20-iris-512.dtb
Tegra20 (Colibri) # set bootargs console=ttyS0,115200n8 earlyprintk
		root=/dev/mmcblk0p1 rootfstype=ext2 rw rootdelay=1
Tegra20 (Colibri) # bootm ${kernel_addr_r} - ${fdt_addr_r}
\endtt





